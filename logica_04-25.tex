\documentclass[a4paper,11pt]{article}
\usepackage[utf8]{inputenc}
\usepackage[spanish]{babel}
\usepackage{amsmath,amsfonts,amsthm,amssymb,enumitem,tikz,tikz-cd,forest,multicol}

\theoremstyle{definition}
\newtheorem{defn}{Definición}[section]
\newtheorem{exap}{Ejemplo}[section]
\newtheorem{prop}{Proposición}[section]
\newtheorem{coro}{Corolario}[section]
\newtheorem{teor}{Teorema}[section]

\newtheorem*{prob}{Problema}
\newtheorem*{ejer}{Ejercicio}

\theoremstyle{remark}
\newtheorem*{remk}{Observación}
\newtheorem*{notc}{Notación}
\newtheorem*{demo}{Demostración}
\newtheorem*{prue}{Prueba}

\def\FF{\ensuremath{\mathcal{F}}}
\def\NN{\mathbb{N}}
\def\FFa{\ensuremath{\mathcal{F}_{\alpha}}}

\title{Lógica y Computabilidad}
\author{XXX}
\date{1er Cuatrimestre 2018}

\begin{document}
\maketitle

\section{Árboles de refutación}

\begin{multicols}{2}
	\begin{enumerate}
		\item $R_{\lnot} = \dfrac{\lnot\lnot\alpha}{\alpha}$
		
		\item $R_{\lnot\vee} = \dfrac{\lnot(\alpha \vee \beta)}{\lnot\alpha , \lnot\beta}$
		
		\item $R_{\wedge} = \dfrac{\alpha \wedge \beta}{\alpha , \beta}$
		
		\item $R_{\lnot\Rightarrow} = \dfrac{\lnot(\alpha \Rightarrow \beta)}{\alpha , \lnot\beta}$
		
		\item $R_{\vee} = \dfrac{(\alpha \vee \beta)}{\alpha \mathop{|} \beta}$
		
		\item $R_{\lnot\wedge} = \dfrac{\lnot(\alpha \wedge \beta)}{\lnot\alpha \mathop{|} \lnot\beta}$
		
		\item $R_{\Rightarrow} = \dfrac{(\alpha \Rightarrow \beta)}{\lnot\alpha \mathop{|} \beta}$
		
		\item[]
	\end{enumerate}
\end{multicols}

\begin{defn}
	Sea $A$ un árbol cuyos elementos son fórmulas. Diremos que un árbol $B$ es una \textbf{extensión inmediata} 
	de $A$ si se verifica las siguientes condiciones:
	
	Si $T$ es un nodo terminal de $A$ y $\alpha$ es un antecesor de $T$ que no es una variable proposicional
	y tampoco la negación de una variable proposicional se tienen dos casos:
	
	\begin{enumerate}
		\item $\alpha$ es una fórmula del tipo 1., 2., 3. ó 4. se agrega como nuevo nodo terminal $\beta$
		si es 1. ó se agrega cualquiera de las conclusiones
		
		\begin{forest}
			[$\alpha$
				[$\cdot$]
				[$\alpha$ 
					[$\alpha$
						[$\beta$,edge=dotted]
					]
					[$\cdot$]
				]
				[$\cdot$]
			]
		\end{forest}
		
		\item Si $\alpha$ es del tipo 5., 6. ó 7. se agregan como nuevos terminales las dos conclusiones
		de las reglas.
		
		\begin{forest}
			[$\cdot$
				[$\cdot$]
				[$\cdot$ 
					[$\cdot$
						[$\alpha_1$,edge=dotted]
						[$\alpha_2$,edge=dotted]
					]
					[$\cdot$]
				]
				[$\cdot$]
			]
		\end{forest}
	\end{enumerate}
\end{defn}

\begin{defn}
	Sea $\alpha$ una fórmula. Un \textbf{árbol de refutación} de $\alpha$ es un árbol $A$ que se obtiene a partir
	del árbol $A_0$ cuyo único nodo es $\alpha$ tal que $A = A_n$ y $A_j$ es una extensión inmediata de 
	$A_{j-1}\ \forall j \in \{1, 2, \dots, n\}$
	
	\[
		A_0 = \{\alpha\} \subseteq A_1 \subseteq \dots \subseteq A_n = A
	\]
\end{defn}

\begin{exap}
	\begin{enumerate}
		\item $\alpha = (P_1 \Rightarrow (P_2 \Rightarrow \lnot\lnot P_3))$ $A = \{\alpha\}$
		
		\item $\alpha$ como en 1. $A$ es el siguiente árbol
		
		\begin{forest}
			[$\alpha$
				[$\lnot P_1$]
				[$(P_1 \Rightarrow P_2)$]
			]
		\end{forest}
		
		\item.
		
		\begin{forest}
			[$\alpha$
				[$\lnot P_1$]
				[$(P_1 \Rightarrow P_2)$
					[$\lnot P_1$]
					[$P_2$]
				]
			]
		\end{forest}

		\item.
		
		\begin{forest}
			[$\alpha$
				[$\lnot P_1$]
				[$(P_1 \Rightarrow P_2)$
					[$\lnot P_1$
						[$\lnot P_1$]
						[$P_2$]
					]
					[$P_2$]
				]
			]
		\end{forest}
	\end{enumerate}
\end{exap}

\begin{defn}
	Si $A$ es un árbol una \textbf{rama} $R$ es un subconjunto de $A$ que verifica las siguientes condiciones:
	
	\begin{enumerate}
		\item $R$ es totalmente ordenado con el orden heredado de $A$.
		
		\item Si $R \subseteq S$ y $S$ es totalmente ordenado entonces $S = R$. Es decir $R$ es un conjunto
		totalmente ordenado y maximal
	\end{enumerate}
\end{defn}

\begin{defn}
	Un conjunto $S$ de fórmulas se dice \textbf{saturado} si verifica las siguientes condiciones:
	
	\begin{enumerate}
		\item Si $\alpha \in F$ entonces $\alpha \not\in S$ ó $\lnot\alpha \not\in S$.
		
		\item 
			\begin{enumerate}
				\item 
				Si $\alpha$ es una fórmula que es premisa de las reglas 1., 2., 3. ó 4. las dos conclusiones
				de las reglas 2., 3. y 4. deben perteneder a $S$ si las premisas estan en $S$ y 
				$\lnot\lnot\alpha \in S$ entonces $\alpha \in S$.
				
				\item Si $\alpha$ es una fórmula en $S$ que es premisa de las reglas 5., 6. ó 7. algunas
				de sus conclusiones debe estar en $S$.
			\end{enumerate}
	\end{enumerate}
\end{defn}

\begin{exap}
	\begin{itemize}
		\item Si $(\alpha \vee \beta) \in S$ entonces $\alpha \in S$ ó $\beta \in S$.
		
		\item Si $(\alpha \wedge \beta) \in S$ entonces $\alpha \in S$ y $\beta \in S$.
		
		\item Si $\lnot(P_1 \Rightarrow \lnot\lnot P_2) \in S$ entonces $P_1 \in S$ y $\lnot\lnot\lnot P_2 \in S$,
		$P_1  \in S$ y $\lnot P_2 \in S$.
	\end{itemize}
\end{exap}

\begin{prop}
	Si $S$ es un conjunto saturado de fórmulas entonces $S$ es satisfactible.
\end{prop}

\begin{demo}
	Sea $v$ la valuación dada por la siguiente fórmula: 
	
	Sea $P_n \in Var$
	
	\[
		v(P_N) = \begin{cases}
			1 & \text{si } P_n \in S \\
			0 & \text{si } P_n \not\in S
		\end{cases}
	\]
	
	Veamos que $v(\alpha) = 1\ \forall \alpha \in S$.
	
	Usaremos inducción en $cb(\alpha)$ ``complejidad binaria de $\alpha$''.
	
	\begin{itemize}
		\item Si $cb(\alpha) = 0$, $\alpha$ es del tipo 
		$\alpha = \underbrace{\lnot\lnot\dots\lnot}_{n \text{veces}} P_i$ con $P_i \in Var$.
		
		Si $n$ es par entonces $P_i \in S$ pues $S$ es saturado. Luego $v(\alpha) = v(P_i) = 1$.
		
		Si $n$ es impar $\lnot P_i \in S$ pues $S$ es saturado. Luego $v(\alpha) = 
		v(\lnot P_i) = 1 - v(P_i) = 1 - 0 = 1$ pues $P_i \not\in S$.
		
		\item Sea $n = cb(\alpha)$ con $n > 0$, supongamos que $v(\beta) = 1\ \forall \beta \in S$
		si $cb(\beta) < n$.
		
		Si $\alpha = \underbrace{\lnot\lnot\dots\lnot}_{n\text{ veces}} \beta$
		
		Si $n$ es par entonces $\beta \in S$. Hay tres casos posibles.
		
		\begin{enumerate}[label=\emph{\alph*})]
			\item $\beta = (\alpha_1 \Rightarrow \alpha_2)$

			\item $\beta = (\alpha_1 \wedge \alpha_2)$

			\item $\beta = (\alpha_1 \vee \alpha_2)$
		\end{enumerate}
		
		Si ocurre c) entonces $\alpha_1 \in S$ ó $\alpha_2 \in S$ pues $S$ es saturado
		por la hipótesis inductiva $v(\alpha_1) = 1$ ó $v(\alpha_2) = 1$ y luego $v(\beta) = 1$.
		
		Si ocurre b) la prueba es análoga.
		
		Si ocurre a) entonces $\lnot \alpha_1 \in S$ ó $\alpha_2 \in S$. Por hipótesis inductiva
		$v(\lnot \alpha_1) = 1$ ó $v(\alpha_2) = 1$ lo que implica $v(\beta) = 1$.
		
		Si $n$ es impar $\lnot \beta \in S$ y hay tres casos:
		
		\begin{enumerate}[label=\emph{\alph*})]
			\item $\lnot\beta = \lnot(\alpha_1 \Rightarrow \alpha_2)$
			
			\item $\lnot\beta = \lnot(\alpha_1 \wedge \alpha_2)$
			
			\item $\lnot\beta = \lnot(\alpha_1 \vee \alpha_2)$
		\end{enumerate}
		
		En forma análoga se prueba que $v(\beta) = 1$.
	\end{itemize}
\end{demo}

\begin{defn}
	Sea $\alpha \in F$ y sea $A$ un árbol de refutación de $\alpha$ $R$ es una rama de $A$.
	
	Diremos que $R$ es \textbf{cerrada} si existe $\beta \in F$ tal que $\beta \in R$ y 
	$\lnot\beta \in R$
	
	Diremos que $R$ es \textbf{saturado} si $R$ es un conjunto saturado de fórmulas.
	
	Diremos que $A$ es \textbf{completo} si toda rama de $A$ es cerrada ó saturada.
\end{defn}

\end{document}
