\documentclass[a4paper,11pt]{article}
\usepackage[utf8]{inputenc}
\usepackage[spanish]{babel}
\usepackage{amsmath,amsfonts,amsthm,amssymb,enumitem,tikz,tikz-cd}

\theoremstyle{definition}
\newtheorem{defn}{Definición}[section]
\newtheorem{exap}{Ejemplo}[section]
\newtheorem{prop}{Proposición}[section]
\newtheorem{coro}{Corolario}[section]
\newtheorem{teor}{Teorema}[section]

\newtheorem*{prob}{Problema}
\newtheorem*{ejer}{Ejercicio}

\theoremstyle{remark}
\newtheorem*{remk}{Observación}
\newtheorem*{notc}{Notación}
\newtheorem*{demo}{Demostración}
\newtheorem*{prue}{Prueba}

\def\FF{\mathcal{F}}
\def\NN{\mathbb{N}}
\def\FFa{\mathcal{F}_{\alpha}}

\title{Lógica y Computabilidad}
\author{XXX}
\date{1er Cuatrimestre 2018}

\begin{document}
\maketitle

\section{Conectivos}

\begin{defn}
Un conjunto finito de conectivos $c_1, c_2, \dots, c_k$ se dice adecuado
si satisface las siguientes propiedad: Si para toda fórmula $\alpha$
existe una fórmula $\beta$ tal que:
\begin{enumerate}
\item $\alpha \equiv \beta$
\item Los conectivos de $\beta$ tienen que pertenecer al conjunto 
$\{c_1, c_2, \dots, c_k\}$.
\end{enumerate}
\end{defn}

\begin{exap}
\begin{enumerate}[label=\emph{\alph*})]
\item $\{\neg, \Rightarrow\}$ es adecuado. 

$(\alpha \vee \beta) \equiv (\neg\alpha \Rightarrow \beta)$ y 
$(\alpha \wedge \beta) \equiv \neg(\alpha \Rightarrow \neg\beta)$

\item $\{\Rightarrow\}$ \textbf{no} es adecuado. Ejemplo de fórmulas
$((P_1 \Rightarrow P_2) \Rightarrow P_2)$, 
$(P_2 \Rightarrow (P_1 \Rightarrow P_3))$

Sea $\alpha = (P_1 \wedge \neg P_1)$, $\alpha$ es contradicción. Veamos
que $\alpha \not\equiv \beta$ para toda fórmula $\beta$ cuyo conectivo es
$\Rightarrow$.

Sea $v$ la valuación que manda todas las variables proposicionales al 1,
$v(P_i) = 1\  \forall i$ entonces $v(\alpha) = 0$.

Veamos que $v(\beta) = 1$ para toda fórmula de $\beta$ si $\Rightarrow$
es el único conectivo que aparece en $\beta$.

Por inducción en $c(\beta)$ = ``número de veces que aparece $\Rightarrow$
en $\beta$''.

\begin{itemize}
\item Si $c(\beta) = 0$, $\beta = P_i$ ($P_i \in Var$) por hipótesis
$v(\beta) = v(P_i) = 1$.

\item Si $c(\beta) = n > 0$ entonces $\beta = (\beta_1 \Rightarrow \beta_2)$
con $c(\beta_1) < n$ y $c(\beta_2) < n$. Por hipótesis inductiva
$v(\beta_1) = 1$ y $v(\beta_2) = 1$. Luego $v(\beta) = 1$.
\end{itemize}

Luego $\{\Rightarrow\}$ no es adecuado.

\begin{remk}
La ``disyunción'' se puede expresar ``semanticamente'' usando solamente
``$\Rightarrow$'', $(\alpha \vee \beta) \equiv 
((\alpha \Rightarrow \beta) \Rightarrow \beta)$.
\end{remk}

\begin{prob}
Caracterizar las funciones booleanas que provienen de una fórmula $\beta$
en el que el único conectivo que aparece en $\beta$ es

\begin{tabular}{c|c}
$P_1$ & $P_1$ \\
\hline
0 & 0 \\
1 & 1
\end{tabular}\quad
\begin{tabular}{c|c}
$P_1$ & $P_1 \Rightarrow P_1$ \\
\hline
0 & 1 \\
1 & 1
\end{tabular}\quad
\begin{tabular}{c|c|c}
$P_1$ & $P_2$ & $(P_1 \Rightarrow P_2)$ \\
\hline
0 & 0 & 1\\
1 & 0 & 0\\
0 & 1 & 1\\
1 & 1 & 1
\end{tabular}

\end{prob}

\item $\{\vee, \wedge\}$ no es adecuado.

Si $v$ es la valuación definida por $v(P_i) = 1\ \forall i$ entonces
$v(\beta) = 1$ si $\beta \in F$ tal que $\text{conectivos}(\beta) 
\subset \{\vee, \wedge\}$.

$(P_1 \Rightarrow P_1)$ es tautología

$(\alpha \Rightarrow \alpha)$ es tautología

\begin{tabular}{c|c|c}
$P_1$ & $P_2$ & $(P_1 \wedge (P_2 \vee P_1)) \equiv P_1$ \\
\hline 
0 & 0 & 0 \\
0 & 1 & 0 \\
1 & 0 & 1 \\
1 & 1 & 1
\end{tabular}

Fijamos $n$ variables $P_1, P_2, \dots, P_n$ el número total de tablas 
de verdad es $2^{2^n}$.

Con 1 variable tenemos 4, $P_1$, $\neg P_1$, $(P_1 \vee \neg P_1)$,
$(P_1 \wedge \neg P_1)$

Con $\{\wedge, \vee\}$ $P_1$, $(P_1 \vee  P_1)$, $(P_1 \wedge P_1)$,
$(P_1 \vee (P_1 \wedge P_1))$

Con 2 variables $\{P_1, P_2\}$ ¿cuántas fórmulas no equivalentes se pueden
construir con dos variables cuyos conectivos son $\{\wedge, \vee\}$.
\end{enumerate}
\end{exap}


\end{document}
