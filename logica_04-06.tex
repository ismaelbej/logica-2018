\documentclass[a4paper,11pt]{article}
\usepackage[utf8]{inputenc}
\usepackage[spanish]{babel}
\usepackage{amsmath,amsfonts,amsthm,amssymb,enumitem}

\theoremstyle{definition}
\newtheorem{defn}{Definición}[section]
\newtheorem{exap}{Ejemplo}[section]
\newtheorem{prop}{Proposición}[section]
\newtheorem{coro}{Corolario}[section]
\newtheorem{teor}{Teorema}[section]

\newtheorem*{prob}{Problema}
\newtheorem*{ejer}{Ejercicio}

\theoremstyle{remark}
\newtheorem*{remk}{Observación}
\newtheorem*{notc}{Notación}
\newtheorem*{demo}{Demostración}
\newtheorem*{prue}{Prueba}

\def\FF{\mathcal{F}}
\def\NN{\mathbb{N}}
\def\FFa{\mathcal{F}_{\alpha}}

\title{Lógica y Computabilidad}
\author{XXX}
\date{1er Cuatrimestre 2018}

\begin{document}
\maketitle

\section{Funciones booleanas}

\begin{defn}
Si $\alpha, \beta \in F$ diremos que $\alpha$ y $\beta$ son \textbf{equivalentes} si
$v(\alpha) = v(\beta)$ para toda valuación $v$. Escribimos $\alpha \equiv \beta$ para
expresar que $\alpha$ y $\beta$ son equivalentes.
\end{defn}

\begin{exap}
\begin{enumerate}
\item $\alpha = P_2$, $\beta = \neg \neg P_2$.
\item Si $\alpha$ y $\beta$ son tautologias entonces $\alpha \equiv \beta$.
\item $\alpha =  ((P_1 \Rightarrow P_2) \Rightarrow P_2)$, $\beta = (P_1 \vee P_2)$. 
Más generalmente $((\alpha_1 \Rightarrow \alpha_2) \Rightarrow \alpha_2) 
\equiv (\alpha_1 \vee \alpha_2)$, con $\alpha_1, \alpha_2 \in F$.
\item $(\alpha_1 \wedge \alpha_2) \equiv \neg (\alpha_1 \Rightarrow \neg \alpha_2)$.
\end{enumerate}
\end{exap}

\begin{defn}
Una función \textbf{booleana} es una función $\FF : 2^n \to 2$ para cierto $n \in \NN\ (n \ge 1)$.
($2 = \{0, 1\}$, $2^n = 2 \times 2 \times \dots \times 2$).
\end{defn}

\begin{defn}[Construcción]
Toda fórmula $\alpha$ induce una función booleana que notaremos con $\FFa$.
Sea $Var(\alpha) = \{P_{j_1}, P_{j_2}, \dots, P_{j_k}\}$ con $j_1 < j_2 < \dots < j_k$.

En este caso $\FFa : 2^k \to 2$ y se define como sigue:

Sea $\vec{a} = (a_1, a_2, \dots, a_k) \in 2^k$ , definimos la siguiente función
$g_{\vec{a}} : Var \to \{0, 1\}$ como sigue: $g_{\vec{a}}(P_{j_1}) = a_1. 
g_{\vec{a}}(P_{j_2}) = a_2, \dots, g_{\vec{a}}(P_{j_k}) = a_k$ y $g_{\vec{a}}(P_i) = 0$
si $P_i \not\in Var(\alpha)$.

Ahora $g_{\vec{a}}$ induce una valuación $v_{g_{\vec{a}}} : F \to 2$ que extiende a $g_{\vec{a}}$.

Definimos $\FFa(\vec{a}) = v_{g_{\vec{a}}}(\alpha)$.
\end{defn}

\begin{exap}
$\alpha = ((P_2 \Rightarrow \neg P_3) \wedge P_7$, $k = 3$, $\FFa : 2^3 \to 2$,
$\FFa(1, 0, 1) = 1$
\end{exap}

\begin{teor}
Si $\FF : 2^n \to 2$ es una función booleana existe una fórmula $\alpha$ con $n$ variables 
proposicionales tal que $\FFa = \FF$.
\end{teor}

\begin{exap}
Sea $\FF : 2^3 \to 2$ dada por $\FF(1, 0, 0) = 1$, $\FF(0, 1, 0) = 1$, $\FF(0, 0, 1) = 1$ y 
$\FF(x, y, z) = 0$ en otro caso.

$\alpha_1 = ((P_1 \wedge \neg P_2) \wedge \neg P_3)$

$\alpha_2 = ((\neg P_1 \wedge P_2) \wedge \neg P_3)$

$\alpha_3 = ((\neg P_1 \wedge \neg P_2) \wedge P_3)$

Respuesta: $\alpha = ((\alpha_1 \vee \alpha_2) \vee \alpha_3)$.

Otra solución: $\beta = (\alpha_1 \vee (\alpha_2 \vee \alpha_3))$

$\alpha \neq \beta$ pero $\alpha \equiv \beta$.
\end{exap}

\begin{demo}[Teorema]
Construimos una fórmula $\alpha$ donde $Var(\alpha) = \{P_1, P_2, \dots, P_n\}$

Sea $T = \{\vec{a} \in 2^n : \FF(\vec{a}) = 1 \}$

Hay dos casos:

1) $T = \emptyset$ basta tomar $\alpha = (\dots ((P_1 \wedge \neg P_1) \wedge P_2 ) \dots \wedge P_n)$
En este caso $\FFa = \FF$

2) $T \neq \emptyset$ como $T$ es finito $T = \{\vec{a_1}, \vec{a_2}, \dots, \vec{a_k}\}$. 
Para cada $j \in \{1, \dots, k\}$ definimos la siguiente fórmula 
$\alpha_{\vec{a_j}} = (\pm P_1 \wedge \pm P_2 \wedge \dots \pm P_n)$ (Esto es
una simplificación de la siguiente fórmula 
$(\dots ((\pm P_1 \wedge \pm P_2) \wedge P_3) \dots \pm P_n)$).
Donde $\pm P_k = \begin{cases}
                  P_k & \text{si }(\vec{a_j})_k = 1 \\
                  \neg P_k & \text{si }(\vec{a_j})_k = 0
                 \end{cases}
$

Se define $\alpha = (\dots ((\alpha_{\vec{a_1}} \vee \alpha_{\vec{a_2}}) 
\vee \alpha_{\vec{a_3}})\dots \alpha_{\vec{a_k}})$.

Resulta que $\FFa = \FF. \qed$
\end{demo}

\begin{defn}
La fórmula $\alpha$ construida de este modo se denomina \textbf{forma normal disyuntiva}.
\end{defn}

Si $\alpha = ((P_1 \Rightarrow P_2) \Rightarrow P_3)$ y $\FFa : 2^3 \to 2$ y sea $\beta$
como en el teorema $\FFa = \FF_{\beta}$. ¿Qué relación hay entre $\alpha$ y $\beta$? Respuesta: 
$\alpha \equiv \beta$.

\begin{remk}
Si $\alpha_1$ y $\alpha_2 \in F$ entonces $(\alpha_1 \Rightarrow \alpha_2) 
\equiv (\neg \alpha_1 \vee \alpha_2)$.
\end{remk}

Construcción de $\beta$.

$\alpha \equiv (\neg (P_1 \Rightarrow P_2) \vee P_3))$ y como 
$(P_1 \Rightarrow P_2) \equiv (\neg P_1 \vee P_2)$. Luego 
$\neg (P_1 \Rightarrow P_2) \equiv \neg(\neg P_1 \vee P_2) 
\equiv (\neg\neg P_1 \wedge \neg P_2) \equiv (P_1 \wedge \neg P_2)$.
Luego $\alpha \equiv ((P_1 \wedge \neg P_2) \vee P_3)$.

\begin{remk}
Otra equivalencias: $\gamma = (\gamma \wedge (\beta \vee \neg \beta)) 
\equiv (\gamma \wedge \beta) \vee (\gamma \wedge \neg \beta))$
\end{remk}

$\alpha \equiv ((P_1 \wedge \neg P_2) \wedge P_3) \vee ((P_1 \wedge \neg P_2) \wedge \neg P_3) 
\vee ((P_3 \wedge P_1) \wedge P_2) \vee ((P_3 \wedge P_1) \wedge \neg P_2)
\vee ((P_3 \wedge \neg P_1) \wedge P_2) \vee ((P_3 \wedge \neg P_1) \wedge \neg P_2)$.

\begin{exap}
 Circuitos $P_1 \wedge P_2$, $((P_1 \wedge P_2) \vee P_3)$.
\end{exap}


\end{document}
