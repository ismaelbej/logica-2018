\documentclass[a4paper,11pt]{article}
\usepackage[utf8]{inputenc}
\usepackage[spanish]{babel}
\usepackage{amsmath,amsfonts,amsthm,amssymb,enumitem,tikz,tikz-cd,forest,multicol}

\theoremstyle{definition}
\newtheorem{defn}{Definición}[section]
\newtheorem{exap}{Ejemplo}[section]
\newtheorem{prop}{Proposición}[section]
\newtheorem{coro}{Corolario}[section]
\newtheorem{teor}{Teorema}[section]

\newtheorem*{prob}{Problema}
\newtheorem*{ejer}{Ejercicio}

\theoremstyle{remark}
\newtheorem*{remk}{Observación}
\newtheorem*{notc}{Notación}
\newtheorem*{demo}{Demostración}
\newtheorem*{prue}{Prueba}

\def\FF{\ensuremath{\mathcal{F}}}
\def\NN{\mathbb{N}}
\def\QQ{\mathbb{Q}}
\def\FFa{\ensuremath{\mathcal{F}_{\alpha}}}
\def\LL{\ensuremath{\mathcal{L}}}
\def\PP{\ensuremath{\mathcal{P}}}

\title{Lógica y Computabilidad}
\author{XXX}
\date{1er Cuatrimestre 2018}

\begin{document}
\maketitle

\section{Árbol de refutación (para lógica de primer orden)}

Reglas para los cuantificadores

\begin{itemize}
    \item $R_{\forall} = \dfrac{\forall x \varphi}{\varphi(x/t)}$, $t$
    es un término sin variables.
    
    \item $R_{\lnot \exists} = \dfrac{\lnot \exists \varphi}{\lnot \varphi(x/t)}$
    
    \item $R_{\exists} = \dfrac{\exists x \varphi}{\varphi(x/c)}$,
    $c$ es un símbolo de constante con restricciones.
    
    \item $R_{\lnot \forall} = \dfrac{\lnot \forall x \varphi}{\lnot \varphi(x/c)}$
\end{itemize}

\begin{defn}
    Sea $A$ un árbol cuyos elementos son enunciados de un lenguaje de
    primer orden. Sea $\alpha$ un nodo terminal de $A$. Diremos que un
    árbol $A'$ es una \textbf{extensión inmediata} de $A$ si se verifica
    las siguientes condiciones:
    
    \begin{enumerate}
        \item Si $\alpha'$ es un antecesor de $\alpha$ del tipo 
        \begin{enumerate}
            \item $\alpha' = (\alpha_1 \lor \alpha_2)$
            \item $\alpha' = (\alpha_1 \Rightarrow \alpha_2)$
            \item $\alpha' = \lnot (\alpha_1 \land \alpha_2)$
        \end{enumerate}
        se agregan en cada uno de los dos casos dos nuevos nodos terminales:
        \begin{enumerate}
            \item $\{\alpha_1, \alpha_2\}$
            \item $\{\lnot \alpha_1, \alpha_2\}$
            \item $\{\lnot \alpha_1, \lnot \alpha_2\}$
        \end{enumerate}
    
        \item Si $\alpha' = (\alpha_1 \land \alpha_2)$,
        $\alpha' = \lnot (\alpha_1 \Rightarrow \alpha_2)$,
        $\alpha' = \lnot (\alpha_1 \lor \alpha_2)$. Se agregan como nodo 
        terminal a $\alpha_1$ ó $\alpha_2$, $\alpha_1$ ó $\lnot \alpha_2$,
        $\lnot \alpha_1$ ó $\lnot \alpha_2$ respectivamente.
        
        \item Si $\alpha' = \lnot\lnot \alpha_1$ se agrega como nuevo
        nodo terminal $\alpha_1$.
        
        \item Si $\alpha' = \forall x \varphi$ ó $\alpha' = \lnot \exists x \varphi$
        se agrega como nuevo nodo terminal $\varphi(x/t)$ en el primer caso y
        $\lnot\varphi(x/t)$ en el segundo caso, donde $t$ es un término sin
        variables.
        
        \item Si $\alpha' = \exists x \varphi$ ó $\alpha' = \lnot \forall x \varphi$
        se agrega como nuevo nodo terminal $\varphi(x/c)$ donde $c$ es un
        símbolo de constante que no aparece en ningún enunciado de $A$.
        En el otro caso se agrega como nodo terminal $\lnot \varphi(x/c)$
        $c$ con la misma restricción.
    \end{enumerate}
\end{defn}

\begin{defn}
    Sea $\varphi$ un enunciado de un lenguaje de primer orden. Diremos que
    $A$ es un \textbf{árbol de refutación} de $\varphi$ si existe una sucesión
    $A_0, A_1, \dots, A_n$ de árboles tales que $A_0 = \{\varphi\}$,
    $A_{i+1}$ es extensión inmediata de $A_i$ para $0 \le i \le n - 1$
    y $A_n = A$.
\end{defn}

\begin{remk}
    Las nociones de rama abierta, rama cerrada, árbol abierto, árbol cerrado
    es la misma que para la lógica proposicional cambiando la palabra
    fórmula por enunciado.
\end{remk}

\begin{remk}
    Si $\Gamma = \{\varphi_1, \varphi_2, \dots, \varphi_n \}$ es un conjunto
    finito de enunciados, diremos que $A$ es un árbol de refutación de $\Gamma$
    si y sólo si $A$ es un árbol de refutación de $(\varphi_1 \land \varphi_2
    \land \dots \land \varphi_n)$.
\end{remk}

\begin{defn}
    Si $\LL$ es un lenguaje de primer orden extendemos el lenguaje $\LL$
    agregandole un conjunto numerable $c_1, c_2, \dots, c_n, \dots$
    de nuevos símbolos de constante.
    
    A este nuevo lenguaje se lo denota $\LL^{\PP}$ y denomina lenguaje
    con parámetros.
\end{defn}

\begin{prop}
    Sea $\LL$ un lenguaje de primer orden y sea $\Gamma$ un conjunto finito 
    y satisfactible de enunciados de $\LL$. Si $A$ es un árbol de refutación
    de $\Gamma$ entonces $A$ debe contener una rama abierta. 
\end{prop}

\begin{demo}
    Se hace inducción en $n$ donde $A_0 = \{\varphi\}$, $A_{i+1}$
    extensión inmediata de $A_i$ $\forall 0 \le i \le n - 1$ y se utiliza
    el hecho siguiente si $H$ es un conjunto satisfactible de enunciados
    y $\forall x \varphi \in H$ entonces $H \cup \{ \varphi(x/t) \}$
    es satisfactible con $t$ un término sin variables y un resultado
    análogo vale para los otros conectivos.
\end{demo}

\begin{coro}
    Si $\LL$ es un lenguaje de primer orden y $\varphi$ es un enunciado de
    $\LL$ tal que $\lnot \varphi$ admite un árbol de refutación cerrado.
    Entonces $\varphi$ es universalmente válido.
\end{coro}

\begin{exap}
    \begin{enumerate}
        \item[]
        
        \item Sea $\LL$ un lenguaje de primer orden y sea
        $\varphi = (\forall x_1 \forall x_2 P x_1 x_2 \Rightarrow
        \forall x_2 \forall x_1 P x_1 x_2)$ con $P$ un símbolo de
        predicado binario ($P \in \LL$).
        
        Trabajamos en $\LL^{\PP}$
        
        Sea $A$ el siguiente árbol de $\lnot \varphi$

		\begin{forest}
            [$\lnot \varphi$
                [$\forall x_1 \forall x_2 P x_1 x_2$
                    [$\lnot \forall x_2 \forall x_1 P x_1 x_2$
                        [$\lnot \forall x_1 P x_1 c_1$
                            [$\lnot P c_2 c_1$
                                [$\forall x_2 P c_2 x_2$
                                    [$P c_2 c_1$]
                                ]
                            ]
                        ]
                    ]
                ]
            ]
        \end{forest}
    
        $A$ es cerrado, luego $\varphi$ es universalmente válida.
        
        \item Sea $\LL$ un lenguaje de primer orden con un símbolo de 
        constante $c$ y un símbolo de predicado unario $P$,
        sea $\varphi = (P c \Rightarrow \exists x_1 P x_1)$.
        
        Sea $A$ el siguiente árbol de $\lnot \varphi$:
        
        \begin{forest}
            [$\lnot \varphi$
                [$P c$
                    [$\lnot \exists x_1 P x_1$
                        [$\lnot P c$]
                    ]
                ]
            ]
        \end{forest}

        Si $\LL$ contiene un símbolo de función unario $\FF$ para cada $n \in \NN$
        sea $A_n$ el siguiente árbol de refutación
        
        \begin{forest}
            [$\lnot \varphi$
                [$P c$
                    [$\lnot \exists x_1 P x_1$
                        [$\lnot P \FF c$
                            [$\lnot P \FF \FF c$
                                [$\lnot P \underbrace{\FF \dots \FF}_{n \text{ veces}} c$]
                            ]
                        ]
                    ]
                ]
            ]
        \end{forest}
    \end{enumerate}
\end{exap}

\begin{defn}
    Sea $\LL$ un lenguaje de primer orden y sea $\Gamma$ un conjunto de enunciados
    de $\LL$. Diremos que $\Gamma$ es \textbf{saturado} si verifica las
    siguientes condiciones:
    
    \begin{enumerate}
        \item Si $\varphi \in \Gamma$ entonces $\lnot \varphi \not\in \Gamma$
        
        \item Si $\lnot \lnot \varphi \in \Gamma$ entonces $\varphi \in \Gamma$
        
        \item Si $(\varphi_1 \lor \varphi_2) \in \Gamma$ entonces
        $\varphi_1 \in \Gamma$ ó $\varphi_2 \in \Gamma$

        \item Si $(\varphi_1 \land \varphi_2) \in \Gamma$ entonces
        $\varphi_1 \in \Gamma$ y $\varphi_2 \in \Gamma$

        \item Si $(\varphi_1 \Rightarrow \varphi_2) \in \Gamma$ entonces
        $\lnot \varphi_1 \in \Gamma$ ó $\varphi_2 \in \Gamma$

        \item Si $\lnot (\varphi_1 \lor \varphi_2) \in \Gamma$ entonces
        $\lnot \varphi_1 \in \Gamma$ y $\lnot \varphi_2 \in \Gamma$

        \item Si $\lnot (\varphi_1 \land \varphi_2) \in \Gamma$ entonces
        $\lnot \varphi_1 \in \Gamma$ ó $\lnot \varphi_2 \in \Gamma$

        \item Si $\lnot (\varphi_1 \Rightarrow \varphi_2) \in \Gamma$ entonces
        $\varphi_1 \in \Gamma$ y $\lnot \varphi_2 \in \Gamma$
        
        \item Si $\forall x \varphi \in \Gamma$ entonces
        $\varphi(x/t) \in \Gamma$ para todo término $t$ sin variables.

        \item Si $\lnot \exists x \varphi \in \Gamma$ entonces
        $\lnot \varphi(x/t) \in \Gamma$, $t$ sin variables.

        \item Si $\exists x \varphi \in \Gamma$ entonces
        $\varphi(x/c) \in \Gamma$ para algún símbolo de constante $c$.

        \item Si $\lnot \forall x \varphi \in \Gamma$ entonces
        $\lnot \varphi(x/c) \in \Gamma$ para algún símbolo de constante $c$.
    \end{enumerate}
\end{defn}

\end{document}
