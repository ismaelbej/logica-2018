\documentclass[a4paper,11pt]{article}
\usepackage[utf8]{inputenc}
\usepackage[spanish]{babel}
\usepackage{amsmath,amsfonts,amsthm,amssymb,enumitem,tikz,tikz-cd}

\theoremstyle{definition}
\newtheorem{defn}{Definición}[section]
\newtheorem{exap}{Ejemplo}[section]
\newtheorem{prop}{Proposición}[section]
\newtheorem{coro}{Corolario}[section]
\newtheorem{teor}{Teorema}[section]

\newtheorem*{prob}{Problema}
\newtheorem*{ejer}{Ejercicio}

\theoremstyle{remark}
\newtheorem*{remk}{Observación}
\newtheorem*{notc}{Notación}
\newtheorem*{demo}{Demostración}
\newtheorem*{prue}{Prueba}

\def\FF{\ensuremath{\mathcal{F}}}
\def\NN{\mathbb{N}}
\def\FFa{\ensuremath{\mathcal{F}_{\alpha}}}

\title{Lógica y Computabilidad}
\author{XXX}
\date{1er Cuatrimestre 2018}

\begin{document}
\maketitle

\section{Consecuencia Lógica}

\begin{defn}[Satisfactibilidad]
	Sea $\Gamma \in F$ y sea $v$ una valuación. Diremos que $v$ \textbf{satisface} a $\Gamma$
	si $v(\alpha) = 1\ \forall\alpha \in \Gamma$.
	
	Diremos que $\Gamma$ es \textbf{satisfactible} si existe una valuación $v$ que satisface a $\Gamma$.
\end{defn}

\begin{remk}
	Un conjunto $\Gamma$ se dice \textbf{insatisfactible} si no es satisfactible.
\end{remk}

\begin{exap}
	\begin{enumerate}
		\item $\Gamma = \{P_0, P_1\}$, 
		
		$\Gamma$ es satisfactible pues cualquier valuación $v$
		tal que $v(P_0) = v(P_1) = 1$ satisface a $\Gamma$.
		
		\item $\Gamma = Var = \{P_0, P_1, \dots, P_n, \dots\}$,
		
		$\Gamma$ es satisfactible por una única valuación.
		
		\item $\Gamma = \{(P_0 \wedge \neg P_0)\}$, es insatisfactible.
		
		\item $\Gamma = \{P_0, \neg P_0\}$, es insatisfactible.
		
		\item $\Gamma = \{(P_0 \vee P_1), \neg P_0, \neg P_1\}$, es insatisfactible.
	\end{enumerate}
\end{exap}

\begin{prop}
	Sea $\Gamma \in F$.
	
	\begin{enumerate}
		\item Si $\Gamma$ es satisfactible y $\Gamma' \subseteq \Gamma$ entonces 
		$\Gamma'$ es satisfactible.
		
		\item Si $\Gamma$ es insatisfactible y $\Gamma \subseteq \Gamma'$ entonces 
		$\Gamma'$ es insatisfactible.
		
		\item $\emptyset$ es satisfactible y $F$ es insatisfactible.
	\end{enumerate}
\end{prop}

\begin{demo}
	\begin{enumerate}
		\item Sea $v$ una valuación que satisface a $\Gamma$. Veamos que $v$ satisface a 
		$\Gamma'$. Sea $\alpha \in \Gamma'$ como $\Gamma' \subseteq \Gamma$ entonces
		$\alpha \in \Gamma$ luego $v(\alpha) = 1$.
		
		\item y 3. son consecuencia de 1.
	\end{enumerate}
\end{demo}

\begin{remk}
	Si $v$ es una valuación cualquiera entonces $v$ satisface a $\emptyset$.
\end{remk}

\begin{defn}
	Sea $\Gamma \subseteq F$ y $\alpha \in F$. Diremos que $\alpha$ es una 
	\textbf{consecuencia lógica} de $\Gamma$ si verifica la siguiente condición:
	
	\begin{itemize}
		\item Si $v$ es una valuación que satisface a $\Gamma$ entonces $v(\alpha) = 1$.
	\end{itemize}
\end{defn}

\begin{notc}
	$C(\Gamma)$ al conjunto de la fórmulas que son consecuencia lógica de $\Gamma$.
\end{notc}

\begin{exap}
	\begin{enumerate}
		\item $\Gamma = \{P_0, P_1\}$ y $\alpha = (P_0 \wedge P_1)$.
		
		\item $\Gamma = \{(P_0 \vee P_1), \neg P_0, \neg P_1\}$ y $\alpha = P_3$
		
		Si $\alpha \not\in C(\Gamma)$ entonces existiría una valuación $v$ que satisface a
		$\Gamma$ pero $v(\alpha) = 0$. Esto es imposible pues $\Gamma$ es insatisfactible.
		
		\item En general si $\Gamma$ es insatisfactible $C(\Gamma) = F$.
	\end{enumerate}
\end{exap}

\begin{teor}
	Sea $\Gamma \subseteq F$. Las siguientes condiciones son equivalentes:
	\begin{enumerate}
		\item $\alpha \in C(\Gamma)$,
		
		\item $\Gamma \cup \{\neg\alpha\}$ es insatisfactible.
	\end{enumerate}
\end{teor}

\begin{demo}
	\begin{itemize}
		\item $[1. \Rightarrow 2.]$ Sea $v$ una valuación y supongamos que por el absurdo
		que $v$ satisface a $\Gamma \cup \{\neg\alpha\}$. En particular $v$ satisface a $\Gamma$,
		como $v$ es consecuencia lógica de $\Gamma$ ($\alpha \in C(\Gamma)$), entonces
		$v(\alpha) = 1$ luego $v(\neg\alpha) = 0$ lo que es un absurdo.
		
		Luego $\Gamma \cup \{\neg\alpha\}$ es insatisfactible.
		
		\item $[2. \Rightarrow 1.]$ Sea $v$ una valuación que satisface a $\Gamma$ veamos
		que $v(\alpha) = 1$. Si $v(\alpha) = 0$ entonces $v(\neg\alpha) = 1$. En particular
		$v$ satisface a $\Gamma \cup \{\neg\alpha\}$ lo que es imposible.
	\end{itemize}
\end{demo}

\begin{remk}
	\begin{enumerate}
		\item Si $\alpha, \beta \in F$ entonces $\beta \in C(\{\alpha\})$ si y sólo si 
		$(\alpha \Rightarrow \beta)$ es tautología.
		
		\item Si $\alpha_1, \alpha_2, \dots, \alpha_n, \beta \in F$ entonces 
		$\beta \in C(\{\alpha_1, \alpha_2, \dots, \alpha_n\})$ si y sólo si 
		$((\alpha_1 \wedge \alpha_2 \wedge \dots \wedge \alpha_n) \Rightarrow \beta)$ es
		tautología.
		
		\item Si $\Gamma = Var = \{P_0, P_1, \dots, P_n, \dots\}$, $C(\Gamma) = \Gamma \cup 
		\{(P_0 \wedge P_1)\} \cup \{((P_0 \wedge P_1) \wedge P_2)\}$.
		
		\item $C(\emptyset) = \text{tautología}$.
		
		\item $C(F) = F$.
		
		\item En general $\Gamma \subseteq C(\Gamma)$.
	\end{enumerate}
\end{remk}

\section*{Teorema de compacidad}

\begin{teor}
	Si $\Gamma \subseteq$ y $\alpha \in C(\Gamma)$ entonces existe $\Gamma' \subseteq \Gamma$ tal que
	$\alpha \in C(\Gamma')$ y $\Gamma'$ es finito.
\end{teor}

(Demostración técnica leer apunte en la página web de la materia)

\begin{teor}
	Las siguientes condiciones son equivalentes:
	\begin{enumerate}
		\item Teorema de Compacidad.
		
		\item Si $\Gamma$ es un conjunto insatisfactible entonces existe $\Gamma' \subseteq \Gamma$ 
		finito e insatisfactible.
	\end{enumerate}
\end{teor}

\begin{demo}
	\begin{itemize}
		\item $[1. \Rightarrow 2.]$ Como $\Gamma$ es insatisfactible entonces $C(\Gamma) = F$.
		En particular $(P_1 \wedge \neg P_1) \in C(\Gamma)$ por 1. existe $\Gamma' \subseteq \Gamma$
		tal que es finito y $(P_1 \wedge \neg P_1) \in C(\Gamma')$.  Como $(P_1 \wedge \neg P_1)$
		es contradicción entonces $\Gamma'$ es insatisfactible.
		
		\item $[2. \Rightarrow 1.]$ Sea $\Gamma \subseteq F$ y sea $\alpha \in C(\Gamma)$.
		Entonces $\Gamma \cup \{\neg\alpha\}$ es insatisfactible por 2. existe 
		$\Gamma' \subseteq \Gamma \cup \{\neg\alpha\}$ tal que $\Gamma'$ es finito e
		insatisfactible.
		
		Si $\neg\alpha \in \Gamma'$ entonces $\Gamma' \setminus \{\neg\alpha\} 
		\subseteq \Gamma$ y $\Gamma' = (\Gamma' \setminus \{\neg\alpha\}) \cup \{\neg\alpha\}$.
		Como $\Gamma'$ es insatisfactible entonces $\alpha \in C(\Gamma' \setminus \{\neg\alpha\})$
		con $\Gamma' \setminus \{\neg\alpha\}$ finito.
		
		Si $\neg\alpha \not\in \Gamma'$ entonces $\Gamma' \subseteq \Gamma$. Como $\Gamma'$
		es insatisfactible entonces $\Gamma' \cup \{\neg\alpha\}$ también es insatisfactible
		lo que implica que $\alpha \in C(\Gamma')$ donde $\Gamma' \subseteq \Gamma$ y 
		$\Gamma'$ es finito.
	\end{itemize}
\end{demo}

\end{document}
