\documentclass[a4paper,11pt]{article}
\usepackage[utf8]{inputenc}
\usepackage[spanish]{babel}
\usepackage{amsmath,amsfonts,amsthm,enumitem}

\theoremstyle{definition}
\newtheorem{defn}{Definición}[section]
\newtheorem{exap}{Ejemplo}[section]
\newtheorem{prop}{Proposición}[section]

\newtheorem*{prob}{Problema}

\theoremstyle{remark}
\newtheorem*{remk}{Observación}
\newtheorem*{notc}{Notación}
\newtheorem*{demo}{Demostración}

\title{Lógica y Computabilidad}
\author{XXX}
\date{1er Cuatrimestre 2018}

\begin{document}
\maketitle

\section{Lógica Proposicional}

\begin{defn} Alfabeto

$Alfabeto = Conectivos \cup Variables\ Proposicionales \cup Par\acute entesis$

$Conectivos = \{ \neg, \vee, \wedge, \Rightarrow \}$

$\neg = \text{``negación''}, \vee = \text{``disyunción''}, \wedge = \text{``conjunción''},
\Rightarrow = \text{``implicación''}$.

$Variables\ Proposicionales = \{ P_0, P_1, \dots, P_n, \dots \}$

$Par\acute entesis = \{(,)\}$
\end{defn}

\begin{defn}
Otra alternativa: 

$Alfabeto = Conectivos \cup Par\acute entesis \cup \{P,|\}$

$P_0 = P, P_1 = P|, P_2 = P||, \dots, P_n = P\underbrace{||\dots|}_{n\text{ veces}}, \dots$.
\end{defn}

\section{Fórmulas}

\begin{defn}
Definición informal: Una \textbf{fórmula} es una expresión sobre el alfabeto anterior
que se define inductivamente como sigue:

\begin{enumerate}
\item $P_n$ es fórmula $\forall n \in \mathbb N$
\item Si $\alpha$ es fórmula entonces $\neg\alpha$ es fórmula
\item Si $\alpha$ y $\beta$ son fórmulas entonces $(\alpha \wedge \beta),
(\alpha \vee \beta), (\alpha \Rightarrow \beta)$ es fórmula
\item Una expresión es fórmula si y sólo si se obtiene en un 
número finito de pasos utilizando las reglas 1), 2) y 3)
\end{enumerate}

\end{defn}

\begin{exap}
$P_2, \neg\neg P_2$ son fórmulas
\end{exap}

\begin{defn}
Sea $A$ el alfabeto de la lógica proposicional, una \textbf{cadena de
formación} es una sucesión finita $X_1, X_2, \dots, X_n$ de expresiones
sobre $A$ que satisfacen las siguientes condiciones:

\begin{enumerate}
\item $X_1$ es variable proposicional.
\item Si $i \in \{1\dots n\}$ entonces $X_i$ es una variable proposicional,

ó existe $j < i$ tal que $X_i = \neg X_j$, 

ó existen $j, k < i$ y $\ast \in \{ \vee, \wedge, \Rightarrow \}$ tal que 
$X_i = (X_j \ast X_k)$.
\end{enumerate}
\end{defn}

\begin{defn}
Una \textbf{fórmula} $\alpha$ es una expresión sobre $A$ tal que existe una
cadena de formación $X_1, X_2, \dots, X_n$ donde $X_n = \alpha$.
\end{defn}

\begin{defn}
Las expresiones $X_i$ se llaman \textbf{eslabones}.
\end{defn}

\begin{notc}
Notamos con $\mathcal F$ al conjunto de todas la fórmulas.
\end{notc}

\begin{defn}
$\mathcal F$ se denomina el lenguaje de la lógica proposicional.
\end{defn}

\begin{remk}
\begin{enumerate}
\item Si $X_1, X_2,\dots X_n$ es una cadena de formación de una fórmula $\alpha$. 
Entonces $X_i$ es fórmula $\forall 1 \le i \le n$.
\item La cadena de formación no es única, toda fórmula $\alpha$ admite
infinitas cadenas de formación.
\end{enumerate}
\end{remk}

\begin{exap}
\begin{enumerate}
\item $\alpha = P_2$ son cadenas de formación: a) $X_1 = P_2$ y b) $X_1 = P_3, X_2 = P2$.
\item Sea $\alpha \in \mathcal F$, si $X_1, \dots, X_n=\alpha$ 
para cada variable proposicional $P_0$ podemos construir las siguientes cadenas de formación.

$P_0, X_1, \dots, X_n$, 

$P_0, P_0, X_1, \dots, X_n$.
\end{enumerate}
\end{exap}

\begin{defn}
Sea $\alpha \in \mathcal F$ una cadena de formación se dice 
\textbf{minimal} si satisface la siguiente propiedad:

Si $X_{j_1}, \dots, X_{j_k}$ es subcadena de formación de 
$X_1, \dots, X_n$ de $\alpha$ entonces $k = n$ y 
$X_{j_1} = X_1, \dots , X_{j_k} = X_n$.
\end{defn}

\begin{exap}
\begin{enumerate}
\item $\alpha = \neg\neg\neg P_3$ la única cadena de formación minimal
de $\alpha$ es $P_3, \neg P_3, \neg\neg P_3, \alpha$.
\item $\alpha = (P_1 \Rightarrow (P_2 \Rightarrow P_3))$ son cadenas de
formación minimales las siguientes:
\begin{enumerate}[label=\alph*)]
\item $P_2, P_1, P_3, (P_2 \Rightarrow P_3), \alpha$
\item $P_1, P_2, P_3, (P_2 \Rightarrow P_3), \alpha$
\end{enumerate}
\end{enumerate}
\end{exap}

\begin{prob}[Difícil]
Si $\alpha \in \mathcal F$ determinar el número de cadenas
de formación minimal de $\alpha$.
\end{prob}

\begin{defn}
Si $\alpha \in \mathcal F$ se define la \textbf{complejidad} de $\alpha$
y se nota $c(\alpha)$ como el número de conectivos que aparecen
en $\alpha$ contados tantas veces como aparecen.
\end{defn}

\begin{exap}
Si $\alpha = (P_1 \Rightarrow (P_2 \Rightarrow P_3)), c(\alpha) = 2$
\end{exap}

\begin{defn}[Definición inductiva]
$c(\alpha)$ se puede definir del siguiente modo:

Si $\alpha = P_1$ entonces $c(\alpha) = 0$,

ó $\alpha = \neg \beta$ (con $\beta \in \mathcal F$)
entonces $c(\alpha) = 1 + c(\beta)$,

ó $\alpha = (\alpha_1 \ast \alpha_2)$ (con $\ast \in \{\vee, 
\wedge, \Rightarrow \}$, $\alpha_1, \alpha_2 \in \mathcal F$) entonces 
$c(\alpha) = 1 + c(\alpha_1) + c(\alpha_2)$.
\end{defn}

\begin{defn}
La \textbf{complejidad binaria} de $\alpha$ es el número de conectivos binarios
que aparecen en $\alpha$ contados tantas veces como aparecen.
\end{defn}

\begin{remk}
$c(\alpha) = 0 \iff \alpha\text{ es una variable proposicional}$
\end{remk}

\begin{notc}
$cb(\alpha)$ = ``complejidad binaria''.

$cb(\alpha) = 0$ si y sólo si $\alpha = \underbrace{\neg\neg\dots\neg}_{k\text{ veces}} P_1$.
\end{notc}

\end{document}
