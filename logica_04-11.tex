\documentclass[a4paper,11pt]{article}
\usepackage[utf8]{inputenc}
\usepackage[spanish]{babel}
\usepackage{amsmath,amsfonts,amsthm,amssymb,enumitem,tikz,tikz-cd}

\theoremstyle{definition}
\newtheorem{defn}{Definición}[section]
\newtheorem{exap}{Ejemplo}[section]
\newtheorem{prop}{Proposición}[section]
\newtheorem{coro}{Corolario}[section]
\newtheorem{teor}{Teorema}[section]

\newtheorem*{prob}{Problema}
\newtheorem*{ejer}{Ejercicio}

\theoremstyle{remark}
\newtheorem*{remk}{Observación}
\newtheorem*{notc}{Notación}
\newtheorem*{demo}{Demostración}
\newtheorem*{prue}{Prueba}

\def\FF{\mathcal{F}}
\def\NN{\mathbb{N}}
\def\FFa{\mathcal{F}_{\alpha}}

\title{Lógica y Computabilidad}
\author{XXX}
\date{1er Cuatrimestre 2018}

\begin{document}
\maketitle

\section{Álgebras de Boole}

\begin{coro}
Sean $\alpha, \beta \in F$ tales que $Var(\alpha) = Var(\beta)$. Entonces  $\alpha \equiv \beta$
si y sólo si $\FFa = \FF_{\beta}$.
\end{coro}

\begin{exap}
Sea $\alpha = (P_1 \Rightarrow P_4)$, $Var(\alpha) = \{P_1, P_4\}$, $\FFa(0, 0) = 1$, $\FFa(0, 1) = 1$,
$\FFa(1, 0) = 0$, $\FFa(1, 1) = 1$.

Sea $\beta = (P_1 \Rightarrow P_2)$, $Var(\alpha) = \{P_1, P_2\}$, $\FF_{\beta} = \FFa$.

Sea $v$ la siguiente valuación $v(P_1) = 1, v(P_2) = 0, v(P_4) = 1$ y $v(P_i) = 0$ en otro caso.
Entonces $v(\alpha) = 1$ y $v(\beta) = 0$, luego $\alpha \not\equiv \beta$.
\end{exap}

\begin{coro}
Si fijamos $n$-variables proposicionales, por ejemplo $P_1$, $P_2$, $\dots$, $P_n$, el número total de
funciones booleanas de $n$-variables es $2^{2^n}$.

En particular existen $2^{2^n}$ fórmulas no equivalentes entre sí cuyas variables son $\{P_1, P_2,
\dots, P_n\}$.
\end{coro}

\begin{exap}
Sea $n = 1$ el número de funciones booleanas es 4. Las siguientes 4 fórmulas son no equivalentes
entres si:

$\alpha_1 = P_1$, $\alpha_2 = \neg P_1$, $\alpha_1 = (P_1 \vee \neg P_1)$, 
$\alpha_1 = (P_1 \wedge \neg P_1)$.

Si $\alpha_5 = (P_1 \Rightarrow \neg P_1)$, entonces $\alpha_5 \equiv \alpha_2$.
\end{exap}

\begin{remk}[Resultado más general]
Si $A$ y $B$ son conjuntos finitos con $k$ y $m$ elementos respectivamente el número total
de funciones de $A$ en $B$ es $m^k$.
\end{remk}

\begin{remk}
Si $\alpha, \beta \in F$ la relación $\equiv$ definida por $\alpha \equiv \beta$ si y sólo si
$v(\alpha) = v(\beta)$ para toda valuación $v$ es una relación de equivalencia en $F$.

El conjunto cociente $F/\equiv\ = \{ |\alpha| : \alpha \in F \}$, donde $|\alpha|$ es la clase
de equivalencia de $\alpha$, $|\alpha| = \{ \beta \in F : \beta \equiv \alpha \}$.

$\alpha \in |\alpha|$, $(\alpha \vee \alpha) \in |\alpha|$ , $\neg \neg \alpha \in |\alpha|$.
\end{remk}

El conjunto $F / \equiv$ es infinito $|P_0|, |P_1|, \dots \in F/\equiv$. si $i \neq j\ |P_i| \neq |P_j|$.

$F / \equiv$ es un álgebra de Boole. Ejemplo de álgebra de Bool $2 = \{0, 1\}$.

$F / \equiv$ es un conjunto parcialmente ordenado con el siguiente orden 
$|\alpha| \leq |\beta|$ si y sólo si $(\alpha \Rightarrow \beta)$ es una tautología.
¿Esta bien definida? Sí.

\begin{enumerate}
\item $\leq$ es una relación de orden parcial
\item $|\alpha|$ con $\alpha$ una contradicción es el primer elemento de $F / \equiv$ pues 
$(\alpha \Rightarrow \beta)$ es tautología $\forall \beta \in F$
\item $|\beta|$ con $\beta$ tautología es el último elemento de $F / \equiv$.
\item El supremo entre $|\alpha|$ y $|\beta|$ es $|(\alpha \vee \beta)|$.

Veamos que
\begin{enumerate}
\item $|\alpha| \leq |(\alpha \vee \beta)|$ si y sólo si $(\alpha \Rightarrow (\alpha \vee \beta))$
es tautología. 

Similarmente $|\beta| \leq |(\alpha \vee \beta)|$.
\item Sea $\gamma \in F$ tal que $|\alpha| \leq |\gamma|$ y $|\beta| \leq |\gamma|$, o sea
$(\alpha \Rightarrow \gamma)$ y $(\beta \Rightarrow \gamma)$ son tautologías. Veamos que
$|(\alpha \vee \beta)| \leq \gamma$, o sea que $((\alpha \vee \beta) \Rightarrow \gamma)$
es tautología.

Sea $v$ una valuación. Si $v((\alpha \vee \beta)) = 0$ entonces 
$v(((\alpha \vee \beta) \Rightarrow \gamma))$ es tautología.

Si $v((\alpha \vee \beta)) = 1$ entonces $v(\alpha) = 1$ ó $v(\beta) = 1$, entonces
$v(\gamma) = 1$ en ambos casos.
\end{enumerate}
\item Análogamente el ínfimo entre $|\alpha|$ y $|\beta|$ es $|(\alpha \wedge \beta)|$
\item Llamando 0 al primer elemento de $F/\equiv$ y 1 al último elemento de $F/\equiv$ se
verifica que si $|\alpha| \in F/equiv$ entonces existe un único $|\beta| \in F/equiv$ tal
que $\underbrace{|\alpha| \vee |\beta|}_{\text{supremo}} = 1$ y 
$\underbrace{|\alpha| \wedge |\beta|}_{\text{infimo}} = 0$.

Respuesta: $|\beta| = |\neg \alpha|$.
\end{enumerate}

\begin{defn}
Un \textbf{Álgebra de Boole} es un conjunto $B$ parcialmente ordenado (con un orden $\leq$) que verifica
las condiciones 1) a 6).
\end{defn}

\begin{exap}
\begin{enumerate}
\item $F/\equiv$ es el álgebra de Bool de la lógica proposicional
\item $F_n/\equiv$ es también un álgebra de Boole finita con $2^{2}$ elementos, donde 
$F_n = \{\alpha \in F : Var(\alpha) = \{ P_1, \dots, P_n\} \}$.

Si $n=1$ vimos que $F_1/\equiv\ = \{|P_1|, |\neg P_1|, |(P_1 \vee \neg P_1)|, 
|(P_1 \wedge \neg P_1)|\}$.

\[
\begin{tikzcd}
& \left|(P_1 \vee \neg P_1)\right| \arrow[dl, dash]\arrow[dr, dash] & \\
\left|P_1\right| \arrow[dr, dash] & & \left|\neg P_1\right| \arrow[dl, dash] \\
& \left|(P_1 \wedge \neg P_1)\right|
\end{tikzcd}
\]
\item $2 = \{0, 1\}$ es el álgebra de Boole con 2 elementos.
\item $2^3 = 2 \times 2 \times 2$ es un álgebra de Boole con el siguiente orden parcial.
$(x, y , z) \leq (x', y', z')$ si y sólo si $x \leq x', y \leq y', z \leq z'$.

\[
\begin{tikzcd}
& (0, 1, 1) \arrow[dl, dash] \arrow[rr, dash] \arrow[dd, dash] & & (1, 1, 1) \arrow[dl, dash] \arrow[dd, dash] \\
(0, 0, 1) \arrow[dd, dash] \arrow[rr, dash] & & (1, 0, 1) \arrow[dd, dash] \\
& (0, 1, 0) \arrow[dl, dash] \arrow[rr, dash] & & (1, 1, 0) \arrow[dl, dash] \\
(0, 0, 0) \arrow[rr, dash] & & (1, 0, 0) 
\end{tikzcd}
\]
\item Si $X$ es un conjunto $\mathcal P(X)$ el conjunto de partes de $X$ es un álgebra de Boole
$A \leq B$ si y sólo si $A \subseteq B$ con $A, B \subseteq X$

Si $A, B \in X$ $A \cup B = B \cup A$

Si $x \in A \cup B$ entonces $x \in A$ ó $x \in B$

$p(x) : x \in A$, $q(x) : x \in B$, $(p(x) \vee q(x))$ es verdadera si $p(x)$ ó $q(x)$ es verdadera.
\end{enumerate}
\end{exap}

\begin{defn}
Un conectivo es una ``función booleana'' de $n$-variables, $c : 2^n \to 2$.
\end{defn}

\begin{exap}
\begin{itemize}
\item $c(P_1, P_2, P_3) = ((P_1 \Rightarrow P_2) \Rightarrow P_3)$
\item $\{\neg, \vee, \wedge, \Rightarrow\}$

Lógica clásica con $\{\neg, \vee\}$ conectivos.

$(\alpha \Rightarrow \beta) \equiv (\neg \alpha \vee \beta)$, 
$(\alpha \wedge \beta) \equiv \neg(\neg \alpha \vee \neg\beta)$.
\end{itemize}

\end{exap}

\end{document}
