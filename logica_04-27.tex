\documentclass[a4paper,11pt]{article}
\usepackage[utf8]{inputenc}
\usepackage[spanish]{babel}
\usepackage{amsmath,amsfonts,amsthm,amssymb,enumitem,tikz,tikz-cd,forest,multicol}

\theoremstyle{definition}
\newtheorem{defn}{Definición}[section]
\newtheorem{exap}{Ejemplo}[section]
\newtheorem{prop}{Proposición}[section]
\newtheorem{coro}{Corolario}[section]
\newtheorem{teor}{Teorema}[section]

\newtheorem*{prob}{Problema}
\newtheorem*{ejer}{Ejercicio}

\theoremstyle{remark}
\newtheorem*{remk}{Observación}
\newtheorem*{notc}{Notación}
\newtheorem*{demo}{Demostración}
\newtheorem*{prue}{Prueba}

\def\FF{\ensuremath{\mathcal{F}}}
\def\NN{\mathbb{N}}
\def\FFa{\ensuremath{\mathcal{F}_{\alpha}}}

\title{Lógica y Computabilidad}
\author{XXX}
\date{1er Cuatrimestre 2018}

\begin{document}
\maketitle

\section{Árboles de refutación}

\begin{prop}
  Toda fórmula $\alpha \in F$ admite un árbol de refutación completo (es decir toda rama
  es cerrada o saturada).
\end{prop}

\begin{exap}
 $\alpha = ((P_1 \Rightarrow P_2) \lor (P_3 \land \lnot P_3))$
 
 \begin{center}
  \begin{forest}
    [$\alpha$
      [$(P_1 \Rightarrow P_2)$
	[$\lnot P_1$]
	[$P_2$]
      ]
      [$(P_3 \land \lnot P_3)$
	[$P_3$
	  [$\lnot P_3$]
	]
      ]
    ]
  \end{forest}
 \end{center}
\end{exap}

\begin{demo}[Proposición]
 Usamos inducción en la complejidad binaria de $\alpha$. Sea $cb(\alpha) = n$
 
 \begin{itemize}
  \item Si $n = 0$, $\alpha = \underbrace{\lnot\lnot\dots\lnot}_{k \text{ veces}} P_i$, $P_i \in Var$.
  
  Si $k$ es par un árbol completo es 
  
  \begin{center}
   \begin{forest}
    [$\alpha$
      [$\underbrace{\lnot\lnot\dots\lnot}_{k-2} P_i$
	[$P_i$]
      ]
    ]
   \end{forest}
  \end{center}

  Similarmente si $k$ es impar el nodo terminal es $\lnot P_i$.
  
  \item Supongamos que $n > 0$, $\alpha = \underbrace{\lnot\lnot\dots\lnot}_{k \text{ veces}}\beta$

  \begin{enumerate}[label=\emph{\alph*})]
   \item $\beta = (\alpha_1 \lor \alpha_2)$
   
   \item $\beta = (\alpha_1 \land \alpha_2)$
   
   \item $\beta = (\alpha_2 \Rightarrow \alpha_2)$
  \end{enumerate}

  Si $k$ es par se comienza la construcción con 
  
  \begin{center}
   \begin{forest}
    [$\alpha$
      [$\underbrace{\lnot\lnot\dots\lnot}_{k-2}\beta$
	[$\beta$]
      ]
    ]
   \end{forest}
  \end{center}
  
  Y tenemos

  \begin{center}
    a)
    \begin{forest} baseline
      [$\cdot$
	[$\beta$
	  [$\alpha_1$]
	  [$\alpha_2$]
	]
      ]
    \end{forest}
    \quad\quad\quad
    c)
    \begin{forest} baseline
      [$\cdot$
	[$\beta$
	  [$\alpha_1$]
	  [$\alpha_2$]
	]
      ]
    \end{forest}
  \end{center}
  En los casos a) y c) $cb(\alpha_1) < n$ y $cb(\alpha_2) < n$ por hipótesis inductiva
  $\alpha_1$ y $\alpha_2$ admiten dos árboles $A_1$ y $A_2$ de refutación completa.
  
  Sea $A$ el siguiente árbol
  
  \begin{center}
   \begin{forest}
    [$\alpha$
      [$\dots$
	[$\beta$
	  [$\alpha_1$
	    [$A_1$,draw]
	  ]
	  [$\alpha_2$
	    [$A_2$,draw]
	  ]
	]
      ]
    ]
   \end{forest}
  \end{center}

  Sea $R$ una rama de $A$. Si $R$ es cerrada no hay nada que probar. Si $R$ es abierta
  hay que probar que $R$ es saturada.
  
  Si $\gamma \in R$ hay cuatro casos posibles:
  
  \begin{enumerate}
   \item $\gamma = \underbrace{\lnot\lnot\dots\lnot}_{k\text{ veces}}\beta$
   
   \item $\gamma = \underbrace{\lnot\lnot\dots\lnot}_{j\text{ veces}}\beta$ con $j < k$ y $j$ par
   
   \item $\gamma = \beta = (\alpha_1 \lor \alpha_2)$
   
   \item $\gamma \in R \cap A_1$ ó $\gamma \in R \cap A_2$
  \end{enumerate}

  Si ocurre 1., 2. ó 3. la única conclusión de $\gamma$ pertenece a $R$.
  
  Si ocurre 4. por hipótesis inductiva la conclusión ó conclusiones de $\gamma$
  pertenecen a $R$.
  
  En el caso c) se prueba análogamente.
  
  Para el caso b) realizamos la siguiente construcción

  \begin{center}
   \begin{forest}
    [$\alpha$
      [$\dots$
	[$\beta$
	  [$\alpha_1$,tikz={\node [draw,fit=()(!2)(!1)] {};}
	    [$\cdot$
	      [$A_2$,draw] {node[anchor=east,align=center]{hola};}
	    ]
	    [$\cdot$
	      [$A_2$,draw]
	    ]
	  ]
	]
      ]
    ]
   \end{forest}
  \end{center}

  En cada nodo terminal de $A_1$ agregamos un nuevo nodo terminal $\alpha_2$ y a
  continuación incorporamos el árbol $A_2$.
  
  Sea $R$ es una rama abierta de este árbol y $\gamma \in R$.
  
  Hay cuatro casos:
  
  \begin{enumerate}
   \item $\gamma = \alpha$
   
   \item $\gamma = \underbrace{\lnot\lnot\dots\lnot}_{j \text{ veces}}\beta$, con $j$ par y $j < k$
   
   \item $\gamma = (\alpha_1 \land \alpha_2)$
   
   \item $\gamma \in R \cap A_1$ ó $\gamma \in R \cap A_2$ pero $\gamma \not\in A_1$
  \end{enumerate}
  
  Al igual que los casos a) y c) se demuestra que las conclusiones de $\gamma$ ó una
  de sus conclusiones pertenece a $R$.
  
  Si $k$ es impar la demostración es análoga. 
 \end{itemize}
\end{demo}

\begin{prop}
 Si $\alpha$ es una fórmula satisfactible entonces todo árbol de refutación de $\alpha$
 tiene una rama abierta.
\end{prop}

\begin{demo}
 (Ejercicio)
\end{demo}

\begin{remk}
 $\alpha$ es satisfactible si y sólo si $\alpha$ no es contradicción.
\end{remk}

\begin{remk}
 Si $v$ es una valuación que satisface a $\alpha$ entonces $v$ satisface a $R$ siendo
 $R$ una rama abierta de cualquier árbol de $\alpha$.
\end{remk}

\begin{coro}
 Sea $\alpha \in F$ las siguientes condiciones son equivalentes:
 
 \begin{enumerate}
  \item $\alpha$ es satisfactible
  
  \item Todo árbol de refutación de $\alpha$ tiene una rama abierta
  
  \item Existe un árbol de refutación completo de $\alpha$ con una rama abierta
 \end{enumerate}
\end{coro}

\begin{demo}
 \begin{itemize}
  \item[]
  
  \item[] $[1. \Rightarrow 2.]$ Es la proposición.
  
  \item[] $[2. \Rightarrow 3.]$ Vimos que toda fórmula $\alpha$ admite un árbol de
  refutación completo y por 2. se deduce 3.
  
  \item[] $[3. \Rightarrow 1.]$ Sea $R$ una rama abierta de un árbol de refutación completo de
  $\alpha$. Como es completo $R$ es saturada y $\alpha \in R$ como $R$ es satisfactible
  en particular $\alpha$ es satisfactible.
 \end{itemize}
\end{demo}

\begin{coro}
 Sea $\alpha \in F$ las siguientes condiciones son equivalentes
 
 \begin{enumerate}
  \item $\alpha$ es contradicción
  
  \item Existe un árbol de refutación de $\alpha$ con todas sus ramas cerradas.
  
  \item Todo árbol de refutación completo tiene todas sus ramas cerradas.
 \end{enumerate}
\end{coro}

\begin{remk}
 El método de los árboles se puede generalizar a un conjunto finito de fórmulas.
 Si $S = \{\alpha_1, \alpha_2, \dots, \alpha_n\}$. $S$ es satisfactible si y sólo si
 $(\alpha_1 \land \alpha_2 \land \dots \land \alpha_n)$ es satisfactible.
\end{remk}

\begin{defn}
 Si $\Gamma \subseteq F$ y $\alpha \in F$. Diremos que $\alpha$ es deducible de $\Gamma$
 y escribimos $\Gamma \vdash \alpha$ si existen $\gamma_1, \gamma_2, \dots, \gamma_n \in \Gamma$
 tales que $\{\gamma_1, \gamma_2, \dots, \gamma_n, \lnot \alpha\}$ es insatisfactible
 ó equivalentemente $(\gamma_1 \land \gamma_2 \land \dots \land \gamma_1 \land \lnot \alpha)$
 es una contradicción.
\end{defn}

\begin{teor}[Teorema de Completitud]
 $\Gamma \vdash \alpha \iff \alpha \in C(\Gamma)$
\end{teor}

\end{document}
