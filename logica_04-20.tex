\documentclass[a4paper,11pt]{article}
\usepackage[utf8]{inputenc}
\usepackage[spanish]{babel}
\usepackage{amsmath,amsfonts,amsthm,amssymb,enumitem,tikz,tikz-cd,forest}

\theoremstyle{definition}
\newtheorem{defn}{Definición}[section]
\newtheorem{exap}{Ejemplo}[section]
\newtheorem{prop}{Proposición}[section]
\newtheorem{coro}{Corolario}[section]
\newtheorem{teor}{Teorema}[section]

\newtheorem*{prob}{Problema}
\newtheorem*{ejer}{Ejercicio}

\theoremstyle{remark}
\newtheorem*{remk}{Observación}
\newtheorem*{notc}{Notación}
\newtheorem*{demo}{Demostración}
\newtheorem*{prue}{Prueba}

\def\FF{\ensuremath{\mathcal{F}}}
\def\NN{\mathbb{N}}
\def\FFa{\ensuremath{\mathcal{F}_{\alpha}}}

\title{Lógica y Computabilidad}
\author{XXX}
\date{1er Cuatrimestre 2018}

\begin{document}
\maketitle

\section{Método de los árboles}

Sea $\alpha = ((P_1 \Rightarrow P_2) \vee (P_2 \Rightarrow P_1))$. Es fácil ver que $\alpha$ es
tautología. Además $\alpha$ es tautología si y sólo si $\neg\alpha$ es contradicción. 
Donde $\neg\alpha = \neg((P_1 \Rightarrow P_2) \vee (P_2 \Rightarrow P_1))$. 

Si $\neg\alpha$
no fuera una contradicción existiría una valuación $v$ tal que $v(\alpha) = 1$. 
Entonces $v(\neg(P_1 \Rightarrow P_2)) = 1$ y $v(\neg(P_2 \Rightarrow P_1)) = 1$.
Luego $v(P_1) = v(\neg P_2) = 1$ y $v(P_2) = v(\neg P_1) = 1$ lo que no es posible.

\begin{center}
	\begin{forest}
		[$\neg\alpha$
			[$\neg(P_1 \Rightarrow P_2)$
				[$\neg(P_2 \Rightarrow P_1)$
					[$P_1$
						[$\neg P_2$
							[$P_2$
								[$\neg P_1$]
							]
						]
					]
				]
			]
		]
	\end{forest}
\end{center}

Construcción del árbol $\beta =  (P_1 \vee (P_2 \wedge \neg P_1))$.

\begin{center}
	\begin{forest}
		[$\beta$
			[$P_1$]
			[$(P_2 \wedge \neg P_1)$
				[$P_2$
					[$\neg P_1$]
				]
			]
		]
	\end{forest}
\end{center}

\begin{prop}
	Sea $\alpha$ una fórmula y sea $v$ una valuación. Entonces se verifican las siguientes condiciones:
	
	\begin{enumerate}
		\item Si $\alpha = \neg\neg\beta$ con $\beta \in F$ y $v(\alpha) =  1$ entonces $v(\beta) = 1$.
		
		\item Si $\alpha = (\beta_1 \vee \beta_2)$ con $\beta_1, \beta_2 \in F$ y $v(\alpha) = 1$ entonces
		$v(\beta_1) = 1$ ó $v(\beta_2) = 1$.
		
		\item Si $\alpha = \neg(\beta_1 \vee \beta_2)$ con $\beta_1, \beta_2 \in F$ y $v(\alpha) = 1$ entonces
		$v(\neg\beta_1) = v(\neg\beta_2) = 1$.
		
		\item Si $\alpha = (\beta_1 \wedge \beta_2)$ con $\beta_1, \beta_2 \in F$ y $v(\alpha) = 1$ entonces
		$v(\beta_1) = v(\beta_2) = 1$.
		
		\item Si $\alpha = \neg(\beta_1 \wedge \beta_2)$ con $\beta_1, \beta_2 \in F$ y $v(\alpha) = 1$ entonces
		$v(\neg\beta_1) = 1$ ó $v(\neg\beta_2) = 1$.
		
		\item Si $\alpha = (\beta_1 \Rightarrow \beta_2)$ con $\beta_1, \beta_2 \in F$ y $v(\alpha) = 1$ entonces
		$v(\neg\beta_1) = 1$ ó $v(\beta_2) = 1$.
		
		\item Si $\alpha = \neg(\beta_1 \Rightarrow \beta_2)$ con $\beta_1, \beta_2 \in F$ y $v(\alpha) = 1$ entonces
		$v(\beta_1) = v(\neg\beta_2) = 1$.
	\end{enumerate}
\end{prop}

\begin{remk}
	Las condiciones 1. a 7. se pueden expresar cambiando "entonces" por "si y sólo si".
\end{remk}

\begin{defn}
	Un árbol $A$ es un conjunto parcialmente ordenado, siendo ``$\le$'' el orden parcial,
	con primer elemento $a_0$ y tal que $\forall a \in A\ \{x \in A : x \le a\}$ 
	es un conjunto finito y totalmente ordenado con el orden original.
\end{defn}

\begin{exap}
	\begin{itemize}
		\item .

		\begin{tabular}{p{6cm}p{6cm}}
			{\begin{forest}
				[$a_0$
					[$a_1$]
					[$a_2$
						[$a_3$]
						[$a_4$
							[$a_5$]
							[$a_6$]
						]
					]
				]
			\end{forest}}
			&
			{\begin{tabular}{cc}
			$a_0 < a_1$ & $a_0 < a_2$ \\
			$a_2 < a_3$ & $a_2 < a_4$ \\
			$a_4 < a_5$ & $a_4 < a_6$
			\end{tabular}}
		\end{tabular}
		
		$\{x \in A : x \le a_4\} = \{a_4, a_0, a_2\}$ 
		
		\begin{center}
			\begin{forest}
				[$a_0$
					[$a_2$
						[$a_4$]
					]
				]
			\end{forest}
		\end{center}
		
		\item Otro ejemplo $A = \NN$.
		
		\item $A = \mathbb Q$ (racionales) $A$ no es un árbol (porque no hay primer elemento).
		
		\item $A = [0,1] = \{x \in \mathbb R : 0 \le x \le 1\}$, si $a = \frac{1}{2}$
		$\{x \in A : x \le a\}$ no es finito.
		
		\item Sea $\mathcal F : X \to \NN$ biyección. En $X$ se define el siguiente orden
		$x \le y \iff \mathcal F(x) \le \mathcal F(y)$. Entonces $X$ es un árbol.
	\end{itemize}
\end{exap}

\begin{defn}
	Sea $(A,\le)$ un árbol y sea $a \in A$. Si $b \in A$ se dice que $b$ es \textbf{sucesor inmediato}
	de $a$ si $a < b$ y no existe $c \in A$ tal que $a < c < b$.

	En el ejemplo 1. $a_5$ y $a_6$ son sucesores inmediatos de $a_4$.
	
	Si $c \in A$, $c$ es \textbf{predecesor inmediato} de $a$ si $c < a$ y no existe $x \in A$
	tal que $c < x < a$.
	
	Si $a \in A$, $a$ puede no tener sucesor inmediato, en este caso $a$ se denomina
	\textbf{nodo terminal}.
\end{defn}

\begin{remk}
	Si $a \in A$ y $a$ no es el primer elemento entonces $a$ tiene un único predecesor
	inmediato.
	
	Pués $\{x \in A : x \le a\} \neq \emptyset$ pues $a_0 < a$ y es un conjunto finito
	y totalmente ordenado $a_0 \le a_1 \le \dots < a_n < a$ y $a_n$ es el único
	predecesor inmediato de $a$.
\end{remk}

Tomemos el siguiente árbol asociado a la fórmula $\neg((P_1 \Rightarrow P_2) \vee (P_2 \Rightarrow P_1))$

\begin{center}
	\begin{forest}
		[$\neg((P_1 \Rightarrow P_2) \vee (P_2 \Rightarrow P_1))$
			[$\neg(P_1 \Rightarrow P_2)$
				[$\neg(P_2 \Rightarrow P_1)$
					[$P_1$
						[$\neg P_2$
							[$P_2$
								[$\neg P_1$]
							]
						]
					]
				]
			]
		]
	\end{forest}
\end{center}

\end{document}
