\documentclass[a4paper,11pt]{article}
\usepackage[utf8]{inputenc}
\usepackage[spanish]{babel}
\usepackage{amsmath,amsfonts,amsthm,amssymb,enumitem}

\theoremstyle{definition}
\newtheorem{defn}{Definición}[section]
\newtheorem{exap}{Ejemplo}[section]
\newtheorem{prop}{Proposición}[section]
\newtheorem{coro}{Corolario}[section]
\newtheorem{teor}{Teorema}[section]

\newtheorem*{prob}{Problema}
\newtheorem*{ejer}{Ejercicio}

\theoremstyle{remark}
\newtheorem*{remk}{Observación}
\newtheorem*{notc}{Notación}
\newtheorem*{demo}{Demostración}
\newtheorem*{prue}{Prueba}

\def\FF{\mathcal{F}}

\title{Lógica y Computabilidad}
\author{XXX}
\date{1er Cuatrimestre 2018}

\begin{document}
\maketitle

\section{Interpretación de las Fórmulas}

\begin{defn}
Una \textbf{valuación} es una función $v : F \to \{0, 1\}$, siendo $F$ el conjunto de todas
las fórmulas, que verifica las siguientes condiciones:

\begin{enumerate}[label=\emph{\alph*})]
\item Si $\alpha \in F$  entonces $v(\neg \alpha) = 1 - v(\alpha)$.
\item Si $\alpha, \beta \in F$ entonces $v((\alpha\vee\beta)) = \max\{v(\alpha), v(\beta)\}$.
\item Si $\alpha, \beta \in F$ entonces $v((\alpha\wedge\beta)) = \min\{v(\alpha), v(\beta)\}$.
\item Si $\alpha, \beta \in F$ entonces $v((\alpha\Rightarrow\beta)) = \max\{1 - v(\alpha), v(\beta)\}$.
\end{enumerate}
\end{defn}

\begin{teor}
Sea $Var$ el conjunto de las variables proposicionales y sea $\mathcal F : Var \to \{0, 1\}$. Entonces
existe una única valuación $v_{\mathcal F} : F \to \{0, 1\}$ que extiende a $\mathcal F$ es decir
$\mathcal F(P_i) = v_{\mathcal F}(P_i)\ \forall i \in \mathbb N$.
\end{teor}

\begin{exap}
Sea $\mathcal F : Var \to \{0, 1\}$ la siguiente función 
\[\mathcal F(P_i) = \begin{cases}
1 & i \text{ es par} \\
0 & i \text{ es impar}
\end{cases} \]
Sea $\alpha = ((P_1 \Rightarrow P_2) \Rightarrow P_3)$, como $\mathcal F(P_1) = 0, \mathcal F(P_2) = 1, 
\mathcal F(P_3) = 0$, entonces $v_{\mathcal F}(\alpha) = 0$.
\end{exap}

\begin{demo}[Teorema]
1) Unicidad: 

Sean $v_1$ y $v_2$ dos valuaciones que extienden a $\mathcal F$. Veamos que $v_1 = v_2$.

Sea $\alpha \in F$ debemos ver que $v_1(\alpha) = v_2(\alpha)$. Sea $X_1, X_2, \dots, X_n$ una 
cadena de formación de $\alpha$. Basta ver que $v_1(X_i) = v_2(X_i)$ para $1 \le i \le n$.

Usando inducción:
\begin{itemize}
\item Si $i = 1$ $X_i$ es una variable proposicional $P_1$. Por hipótesis
$v_1(P_1) = \mathcal F(P_1) = v_2(P_1)$.
\item Sea $1 < i \le n$ y supongamos que $v_1(X_j) = v_2(X_j)$ si $j < i$ (hipótesis inductiva).
Vamos a probar que $v_1(X_i) = v_2(X_i)$

\begin{enumerate}[label=\emph{\roman*})]
\item Si $X_i$ es variable proposicional se prueba como en el caso base.
\item Si $X_i = \neg X_j$ con $j < i$ tenemos $v_1(X_i) = v_1(\neg X_j) = 1 - v_1(X_j)$.
Similarmente $v_2(X_i) = v_2(\neg X_j) = 1 - v_2(X_j)$. Por hipótesis inductiva
$v_1(X_j) = v_2(X_j)$ lo que implica $v_1(X_i) = v_2(X_i)$.
\item Si $X_i = (X_j \vee X_k)$ con $j, k < i$, vale que $v_1(X_i) = \max\{v_1(X_j), v_1(X_k)\}$
y $v_2(X_i) = \max\{v_2(X_j), v_2(X_k)\}$, por hipótesis inductiva $v_1(X_j) = v_2(X_j)$ y
$v_1(X_k) = v_2(X_k)$ lo que implica $v_1(X_i) = v_2(X_i)$
\end{enumerate}
Los demás casos $X_i = (X_j \wedge X_k)$ y $X_i = (X_j \Rightarrow X_k)$ con $j, k < i$
se prueban análogamente.
\end{itemize}

2) Existencia:

Queremos calcular el valor de $v_{\mathcal F}(\alpha)\ \forall \alpha \in F$.

Haciendo inducción en $c(\alpha) = n$:

\begin{enumerate}
\item Si $n=0$, $\alpha \in Var$, $\alpha = P_i$ y luego definimos 
$v_{\mathcal F}(\alpha) = \mathcal F(P_i)$.
\item Supongamos que hemos definido $v_{\FF}(\beta)$ si $c(\beta) < n$ (hipótesis inductiva).

Si $\alpha \in F$ y $c(\alpha) = n,\ (n > 0)$, tenemos los siguientes casos.
\begin{enumerate}[label=\emph{\alph*})]
\item $\alpha = \neg\beta$
\item $\alpha = (\beta_1 \vee \beta_2)$
\item $\alpha = (\beta_1 \wedge \beta_2)$
\item $\alpha = (\beta_1 \Rightarrow \beta_2)$
\end{enumerate}
con $\beta, \beta_1, \beta_2$ fórmulas.

En el caso a) se define $v_{\FF}(\alpha) = 1 - v_{\FF}(\beta)$, donde
$v_{\FF}(\beta)$ es conocido por hipótesis inductiva.

En los casos b), c) y d) se define $v_{\FF}(\alpha) = \max\{v_{\FF}(\beta_1), v_{\FF}(\beta_2)\}$,
$v_{\FF}(\alpha) = \min\{v_{\FF}(\beta_1), v_{\FF}(\beta_2)\}$ y
$v_{\FF}(\alpha) = \max\{1 - v_{\FF}(\beta_1), v_{\FF}(\beta_2)\}$ donde
$v_{\FF}(\beta_1)$ y $v_{\FF}(\beta_2)$ son conocidos por hipótesis inductiva.
\end{enumerate}
\end{demo}

\begin{teor}
Sea $\alpha \in F$ y sea $Var(\alpha)$ el conjunto de variables proposicionales que 
aparecen en $\alpha$. Si $v_1$ y $v_2$ son dos valuaciones que coinciden en $Var(\alpha)$
entonces $v_1(\alpha) = v_2(\alpha)$
\end{teor}

(Prueba como ejercicio)

\begin{exap}
Dado $\alpha = (P_1 \vee (P_2 \Rightarrow P_3))$, $Var(\alpha) = \{P_1, P_2, P_3\}$. 
Si $v_1(P_1) = v_2(P_1) = 0, v_1(P_2) = v_2(P_2) = 1, v_1(P_3) = v_2(P_3) = 0$
y $v_1(P_4) = 1, v_2(P_4) = 0$, entonces $v_1(\alpha) = v_2(\alpha) = 0$.
\end{exap}

\begin{defn}
Sea $\alpha \in F$ diremos que $\alpha$ es una \textbf{tautología} si $v(\alpha) = 1$
para toda valuación $v$. Diremos que $\alpha$ es una \textbf{contradicción} si 
$v(\alpha) = 0$ para toda valuación $v$. Diremos que $\alpha$ es una \textbf{contingencia}
si no es ni tautología ni contradicción.
\end{defn}

\begin{exap}
\begin{enumerate}
\item $\alpha = (P_1 \vee \neg P_1)$, $\alpha$ es una tautología.
\item $\alpha = (\beta \vee \neg \beta)$, con $\beta \in F$, $\alpha$ es una tautología. 
``Principio del tercer excluido''.
\item Sabemos $2 = \{0, 1\}$, podriamos definir otra lógica $3 = \{0, \frac{1}{2}, 1\}$,
como $\neg x = 1 - x$, entonces $\neg \frac{1}{2} = \frac{1}{2}, \neg 1 = 0, \neg 0 = 1$.
Tenems que $\alpha = (P_1 \vee \neg P_1)$ \emph{no} es una tautología en esta lógica trivalente.
\end{enumerate}
\end{exap}

\begin{remk}
¿Se puede determinar $\alpha$ dada una tabla de verdad? Sí, para la lógica clásica,
pero no es cierto para otras lógicas.
\end{remk}


\end{document}
