\documentclass[a4paper,11pt]{article}
\usepackage[utf8]{inputenc}
\usepackage[spanish]{babel}
\usepackage{amsmath,amsfonts,amsthm,amssymb,enumitem,tikz,tikz-cd,forest,multicol}

\theoremstyle{definition}
\newtheorem{defn}{Definición}[section]
\newtheorem{exap}{Ejemplo}[section]
\newtheorem{prop}{Proposición}[section]
\newtheorem{coro}{Corolario}[section]
\newtheorem{teor}{Teorema}[section]

\newtheorem*{prob}{Problema}
\newtheorem*{ejer}{Ejercicio}

\theoremstyle{remark}
\newtheorem*{remk}{Observación}
\newtheorem*{notc}{Notación}
\newtheorem*{demo}{Demostración}
\newtheorem*{prue}{Prueba}

\def\FF{\ensuremath{\mathcal{F}}}
\def\NN{\mathbb{N}}
\def\QQ{\mathbb{Q}}
\def\FFa{\ensuremath{\mathcal{F}_{\alpha}}}
\def\LL{\ensuremath{\mathcal{L}}}
\def\PP{\ensuremath{\mathcal{P}}}

\title{Lógica y Computabilidad}
\author{XXX}
\date{1er Cuatrimestre 2018}

\begin{document}
\maketitle

\section{Sustitución de variables}

\begin{defn}
    Sea $\LL$ un lenguaje de primer orden y sea $\varphi$ en enunciado de $\LL$. 
    Diremos que una interpretación $I$ \textbf{satisface} a $\varphi$
    si $v_I(\varphi) = 1$.

    Diremos que $\varphi$ es \textbf{satisfacible} si existe una
    interpretación $I$ que satisface a $\varphi$. Caso contrario diremos
    que $\varphi$ es \textbf{insatisfacible}.

    Diremos que $\varphi$ es \textbf{universalmente válido} si
    $v_I(\varphi) = 1$ para toda interpretación $I$.
\end{defn}

\begin{remk}
    $\varphi$ es universalmente válida si y sólo si $\lnot \varphi$
    es insatisfacible.
\end{remk}

\begin{exap}
    Sea $\LL$ un lenguaje que consiste de un símbolo de predicado
    binario $P$

    \begin{enumerate}
        \item Sea $\varphi = \underbrace{\forall x_1 \forall x_2 
        \underbrace{(\overbrace{P\,x_1\,x_2}^{\text{fórmula}} \lor 
        \lnot \overbrace{P\, x_1\,x_2)}^{\text{fórmula}}}_{\text{fórmula}}}_{\text{enunciado}}$

        \item Sea $\gamma = (\forall x_1 \forall x_2 P x_1 x_2
        \Rightarrow \forall x_2 \forall x_1 P x_1 x_2)$

        Veamos que $v_I(\gamma) = 1$ para toda interpretación $I$. 
        Sea $I = (U, P_U)$ con $P_U \subseteq U\times U$.

        Supongamos por el absurdo que $v_I(\gamma) = 0$, luego
        $v_I(\forall x_1 \forall x_2 P x_1 x_2) = 1$ y 
        $v_I(\forall x_2 \forall x_1 P x_1 x_2) = 0$.
        e $v_I(\forall x_1 P x_1 c_a) = 0$
        y existe $b \in U$ tal que $v_I(P c_b c_a) = 0$ si y sólo si
        $(b, a) \not\in P_U$. Como $v_I(\forall x_1 \forall x_2 P x_1 x_2) = 1$
        entonces $v_I(P c_b c_a) = 1$. Absurdo.
    \end{enumerate}
\end{exap}

\begin{defn}
    Sea $\LL$ un lenguaje de primer orden y sea $\Gamma$ un conjunto de
    enunciados. Diremos que una interpretación $I$ \textbf{satisface}
    a $\Gamma$ o que $I$ es un \textbf{modelo} de $\Gamma$ si
    $v_I(\varphi) = 1 \forall \varphi \in \Gamma$.
    
    Diremos que $\Gamma$ es \textbf{satisfacible} si admite un modelo.
    
    Si $\Gamma$ no es satisfacible diremos que $\Gamma$ es
    \textbf{insatisfacible}.
    
    Si $\alpha$ es un enunciado diremos que es \textbf{consecuencia} de $\Gamma$
    si $v_I(\alpha) = 1$ para toda interpretación $I$ que satisface a $\Gamma$.
\end{defn}

\begin{defn}
    Un lenguaje $\LL$ de primer orden se dice un lenguaje con igualdad
    si $\LL$ contiene un símbolo de predicado binario ``$=$'' en el que
    se interpreta con la relación de identidad o igualdad.
\end{defn}

\begin{exap}
    \begin{enumerate}
        \item Sea $\LL$ un lenguaje con igualdad y un conjunto numerable 
        $c_1, c_2, \dots, c_n, \dots$ de símbolos de constante.
     
        Sea $\Gamma$ el siguiente conjunto de enunciados:
        \[\Gamma = \{ 
            \lnot c_1 = c_2, \lnot c_1 = c_3, \lnot c_2 = c_3, \dots
            \lnot c_i = c_j\ i\neq j\ i,j \in \NN
        \}\]
     
        ¿Es $\Gamma$ satisfacible? Sí, 
        $I = \langle \NN, =, c_{i \NN} = i\rangle$.
     
        \begin{remk}
            Toda interpretación $I$ que satisface a $\Gamma$ debe
            tener un universo infinito.
        \end{remk}
     
        \item Sea $\LL$ un lenguaje con un símbolo de predicado binario $P$.
        
        Sea $\Gamma = \{\alpha_1, \alpha_2, \alpha_3\}$.
        
        $\alpha_1 = \forall x_1 \lnot P x_1 x_1$,
        $\alpha_2 = \forall x_1 \exists x_2 P x_1 x_2$,
        $\alpha_3 = (\forall x_1 \forall x_2 P x_1 x_2
            \Rightarrow \exists x_3 (P x_1 x_3 \land P x_3 x_2))$
            
        La siguiente interpretación satisface a $\Gamma$, 
        $I = \langle \QQ, \leq \rangle$, con $\QQ$ racionales.
        
        ¿$\Gamma$ admite modelos finitos? 
        
        \begin{itemize}
            \item Si $\{a\} = U$ por $\alpha_1$ y $\alpha_2$ no puede ser.
            
            \item Si $\{a, b\} = U$ por $\alpha_2$ $P a b$
            por $\alpha_3$ existe $x_3$ tal que $P a x_3$ y $P x_3 b$
            pero por $\alpha_1$ no pude ser ni $a$ ni $b$.
        \end{itemize}
        
        No. $U$ tiene que ser infinito.
    \end{enumerate}
\end{exap}

\section*{Expresividad de los lenguajes de primer orden}

\begin{exap}
    Sea $\LL$ un lenguaje con igualdad. Sea 
    $\alpha_1 = \exists x_1 \exists x_2 \lnot x_1 = x_2$
    
    $\alpha_1$ expresa que un conjunto tiene al menos 2 elementos.
    
    $\alpha_2 = \exists x_1 \exists x_2 (\lnot x_1 = x_2 \land 
    \forall x_3 (x_3 = x_1 \lor x_3 = x_2))$.
    
    $\alpha_2$ expresa que un conjunto tiene exactamente 2 elementos.
\end{exap}

\begin{ejer}
    Escribir un enunciado $\alpha$ de $\LL$ que exprese
    
    \begin{enumerate}[label=\emph{\alph*})]
        \item Un conjunto tiene al menos $n$ elementos
        
        \item Un conjunto tiene a lo sumo $n$ elementos
        
        \item Un conjunto tiene exactamente $n$ elementos
    \end{enumerate}
\end{ejer}

\begin{remk}
    No se puede expresar en $\LL$ que un conjunto es finito.
    
    Más precisamente:

    \begin{enumerate}[label=\emph{\alph*})]
        \item No existe un enunciado $\alpha$ de $\LL$ tal que
        $v_I(\alpha) = 1$ si y sólo si $I$ es finito.
        
        \item No existe ningún conjunto $\Gamma$ de enunciados
        de $\LL$ tal que $I$ satisface a $\Gamma$ si y sólo si
        el universo de $I$ es finito.
    \end{enumerate}

    La prueba requiere el uso del teorema de compacidad.
    
    Sea para cada $n \in \NN$ un enunciado $\alpha_n$ que
    exprese que un conjunto tiene al menos $n$ elementos.
    
    Sea $\Gamma = \{ \alpha_1, \alpha_2, \dots,\alpha_n, \dots \}$.
    
    Una interpretación $I$ es modelo de $\Gamma$ si y sólo si
    el universo de $I$ es infinito.
\end{remk}

\end{document}
