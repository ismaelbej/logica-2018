\documentclass[a4paper,11pt]{article}
\usepackage[utf8]{inputenc}
\usepackage[spanish]{babel}
\usepackage{amsmath,amsfonts,amsthm}

\theoremstyle{definition}
\newtheorem{defn}{Definición}[section]
\newtheorem{exap}{Ejemplo}[section]
\newtheorem{prop}{Proposición}[section]

\theoremstyle{remark}
\newtheorem*{remk}{Observación}
\newtheorem*{notc}{Notación}
\newtheorem*{demo}{Demostración}

\title{Lógica y Computabilidad}
\author{XXX}
\date{1er Cuatrimestre 2018}

\begin{document}
\maketitle

\section{Preeliminares}

\begin{notc}
El conjunto de los números naturales es el conjunto:

\[\mathbb N = \{ 0, 1, 2, 3, \dots \}\]
\end{notc}

\[
\text{Principio de inducción:}
\begin{cases}
&\text{Versión estándard} \\
&\text{Versión general (principio de inducción completo)}
\end{cases}
\]

\section{Alfabeto y Lenguajes}

\begin{defn}
Un \textbf{alfabeto} es un conjunto $A$ no vacío.
\end{defn}

\begin{defn}
Una \textbf{expresión} sobre un alfabeto $A$ es una sucesión o
secuencia finita sobre $A$. Más precisamente una expresión sobre $A$ es
una sucesión: $a_0, a_1, \dots, a_n$ con $a_i \in A$.
\end{defn}

\begin{remk}
El número $n+1$ se denomina la \textbf{longitud} de la expresión,
\end{remk}

\begin{notc}
$\ell(a_0a_1\dots a_n) = n+1$
\end{notc}

\begin{notc}
Notaremos $A^*$ al conjunto de todas las expresiones sobre $A$
\end{notc}

\begin{exap}
$A = \{0, 1\}$ una expresión sobre $A$ es por ejemplo: $001100$
\end{exap}

\begin{exap}
$A = \{a, b, c, \dots, z\}$ una expresión sobre $A$ es: $bcdf$
\end{exap}

\begin{defn}
Sea $A$ un alfabeto y sean $E, F \in A^*$, la concatenación 
de $E$ con $F$ es la expresión $EF$ que se define del siguiente modo, 
si $E = a_0a_1\dots a_n$ y $F = b_0b_1\dots b_m$ entonces:

\[ EF = a_0a_1\dots a_nb_0b_1\dots b_m \]
\end{defn}

\begin{remk}
$\ell(EF) = \ell(E) + \ell(F)$.
\end{remk}

\begin{notc}
Si $E,F \in A^*$ diremos que $F$ es una \textbf{sección inicial} de $E$ si
existe $G \in A^*$ tal que $E = FG$.
\end{notc}

\begin{remk}
La expresión vacía es un caso particular de expresión.
\end{remk}

\begin{prop}
Si $E, F, G, H \in A^*$ tal que $EF = GH$. Entonces $E$
es una sección inicial de $G$ ó $G$ es una sección inicial de $E$.

Más aún si $\ell(E) \ge \ell(G)$ entonces $G$ es sección inicial de $E$.
\end{prop}

\begin{demo}[Por inducción]
Sea $P(n) =$  ``Si $E \in A^*$ tal que $\ell(E) = n$ y $F, G, H \in A^*$
que verifican $EF = GH$ entonces $E$ es una sección inicial de $G$ si 
$\ell(E) \le \ell(G)$ ó $G$ es una sección inicial de $E$ si 
$\ell(G) \le \ell(E)$''.
\end{demo}

\begin{enumerate}
\item $P(0)$ se cumple trivialmente.
\item $P(n) \Rightarrow P(n+1)$.

Sea $E \in A^*$ con $\ell(E) = n+1$ con 

\[ 
\begin{align*}
E &= e_0e_1\dots e_n \\
F &= f_0f_1\dots f_m \\
G &= g_0g_1\dots g_k \\
H &= h_0h_1\dots h_l
\end{align*}
\]

Como $EF = GH$ entonces $e_0e_1\dots e_nf_0f_1\dots f_m = g_0g_1\dots g_kh_0h_1\dots h_l$
luego $e_0 = g_0$ entonces $\underbrace{e_1\dots e_n}_{E'}f_0f_1\dots f_m 
= \underbrace{g_1\dots g_k}_{G'}h_0h_1\dots h_l$.

Por hipótesis inductiva tenemos $\ell(E') = n$ y luego vale la tesis 
lo que implica $E'$ es sección inicial de $G'$ ó $G'$ es sección
inicial de $E'$.

\[
\begin{align*}
E' &= G'\_ & &\Rightarrow & e_0E' &= g_0 G'\_ & &\iff & E &= G\_\\
G' &= E'\_ & & & g_0 G' &= e_0 E'\_ & & & G &= E\_
\end{align*}
\]

Entonces $E$ es una sección inicial de $G$ ó al revés.
\end{enumerate}

\begin{defn}
Sea $A$ un alfabeto un \textbf{lenguaje} $\mathcal L$ sobre $A$ 
es un subconjunto no vacío de $\mathcal L$ de $A^*$ (expresiones sobre $A$).
\end{defn}

\begin{exap}
$A = \{0, 1, 2, \dots, 9\}$, 

$\mathcal L = \{ E \in A^* : 
E \textrm{ es un número natural} \}$

$E \in \mathcal L \iff \ell(E) = 1$ ó $\ell(E) > 1$ y 
$E = a_0a_1\dots a_n$ con $a_0 \ne 0$.
\end{exap}

\begin{exap}
$A = \{a, b, c, \dots, z\}$, 

$\mathcal L = \left\{ E \in A^* : 
\substack{E \text{ es una palabra que aparece} \\ \text{en el diccionario de la RAE}}\right\}$

$x \in A,\quad x, xx, xxx, \dots, \underbrace{xx\dots x}_{n\textrm{ veces}} 
, \dots\in A^*$ son expresiones pero no estan en el lenguaje $\mathcal L$.

$boda, casamiento \in \mathcal L$, sintácticamente no son iguales,
semánticamente representan lo mismo.
\end{exap}

\begin{exap}
$A = \{A,C,G,T\}$ Alfabeto del ADN

$\mathcal L = \left\{ E \in A^* : \text{E es de la forma } 
ACGT\underbrace{AAA\ CGA\ \_\_\_}_{\text{expr. long. 3}}\dots \substack{TGA \\ TAG \\ TAA}\right\}$
\end{exap}

\begin{exap}
Lenguaje de programación: $v \leftarrow v + 1$, $v \leftarrow v - 1$, if $v \ne 0$ goto A$.
\end{exap}

\end{document}
