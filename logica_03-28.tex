\documentclass[a4paper,11pt]{article}
\usepackage[utf8]{inputenc}
\usepackage[spanish]{babel}
\usepackage{amsmath,amsfonts,amsthm,amssymb,enumitem}

\theoremstyle{definition}
\newtheorem{defn}{Definición}[section]
\newtheorem{exap}{Ejemplo}[section]
\newtheorem{prop}{Proposición}[section]
\newtheorem{coro}{Corolario}[section]

\newtheorem*{prob}{Problema}
\newtheorem*{ejer}{Ejercicio}

\theoremstyle{remark}
\newtheorem*{remk}{Observación}
\newtheorem*{notc}{Notación}
\newtheorem*{demo}{Demostración}
\newtheorem*{prue}{Prueba}

\title{Lógica y Computabilidad}
\author{XXX}
\date{1er Cuatrimestre 2018}

\begin{document}
\maketitle

\section{Unicidad de escritura de Fórmulas}

\begin{defn}
Sea $A$ el alfabeto de la lógica proposicional y sea $E \in A^*$, el \textbf{peso} de $E$ es
el número de paréntesis abiertos ``('' menos el número de paréntesis cerrados ``)'' 
que aparecen en $E$.
\end{defn}

\begin{exap}
$E = (P_1 \Rightarrow ))$ el peso de $E$ es $-1$.
\end{exap}

\begin{notc}
El peso de $E$ se denota $P(E)$.
\end{notc}


\begin{prop}
\begin{enumerate}
\item Si $\alpha$ es una fórmula entonces $P(\alpha) = 0$.
\item Si $\alpha$ es una fórmula y $*$ es un conectivo binario que aparece en $\alpha$
entonces $P(E) > 0$ donde $E$ es la expresión que aparece a la izquierda de $*$.
\end{enumerate}
\end{prop}

\begin{exap}
$\alpha = \underbrace{\overbrace{(P_1}^{E_1} \Rightarrow (P_2}_{E_2} \Rightarrow P_3))$,
$P(E_1) = 1$ y $P(E_2) = 2$
\end{exap}

\begin{prue}[Proposición]

1) Sean $X_1, X_2, \dots, X_n$ una cadena de formación de $\alpha$. Probemos que
$P(X_i) = 0$ para todo $1 \le i \le n$.

\begin{itemize}
\item Caso $P(X_1) = 0$, esto vale pues $X_1 = P_l$ una variable proposicional.
\item Supongamos $i > 0$ y veamos que $P(X_i) = 0$.

Hay 3 subcasos:
\begin{enumerate}[label=\alph*)]
\item $X_i$ es una variable proposicional, luego $P(X_i) = 0$.
\item $X_i = \neg X_j$ con $j < i$, luego $P(X_i) = P(X_j)$. Por hipótesis inductiva
$P(X_j) = 0$. Luego $P(X_i) = 0$.
\item $X_i = (X_j * X_k)$ con $j, k < i$, $* \in \{\vee, \wedge, \Rightarrow\}$.
Por hipótesis inductiva $P(X_j) = P(X_k) = 0$. En todos los casos 
$P(X_i) = 1 + \underbrace{P(X_j)}{=0} + \underbrace{P(X_k)}_{=0} - 1 = 0$.
\end{enumerate}
\end{itemize}

2) Sean $X_1, X_2, \dots, X_n$ una cadena de formación de $\alpha$. Probemos por
inducción en $i$ que $X_i$ verifica la proposición a probar.

\begin{itemize}
\item Si $i = 1$ $X_1$ es una variable proposicional. En este caso la proposición
se verifica trivialmente, pues no hay conectivos binarios en $X_1$.
\item Si $i > 1$ hay tres casos:

\begin{enumerate}[label=\alph*)]
\item $X_i$ es variable proposicional. Ídem caso $i=1$.
\item $X_i = \neg X_j$ con $j < i$. Si $*$ es un conectivo binario en $X_i$
entonces $*$ aparece en $X_j$, luego $X_j = E*\_$. Por hipótesis inductiva
$P(E) > 0$ y como $X_i = \neg E * \_$. Como $P(\neg E) = P(E) > 0$. Entonces
$X_i$ satisface la propiedad.
\item $X_i = (X_j \circ X_k)$ con $j,k < i$, si $*$ es un conectivo binario
que aparece en $X_i$ hay tres subcasos:
\begin{enumerate}[label=\roman*)]
\item $*$ aparece en $X_j$, $X_j = E*\_$, por hipótesis inductiva $P(E) > 0$ la
expresión que aparece a la izquierda de $*$ es $\hat E = (E$. Luego
$P(\hat E) = 1 + \underbrace{P(E)}_{> 0}$ lo que implica $P(\hat E) > 0$.
\item $* = \circ$ la expresión que aparece a la izquierda de $*$ es 
$\hat E = (X_j$ lo que implica $P(\hat E) = 1 + \underbrace{P(X_j)}_{=0} = 1 > 0$.
\item $*$ aparece en $X_k$, $X_k = E*\_$, la expresión que aparece a la 
izquierda de $*$ en $X_i$ es $(X_j \circ E$. Por hipótesis inductiva
$P(E) > 0$ como $P(X_j) = 0$ se concluye que 
$P((X_j\circ E) = 1 + \underbrace{P(X_j)}_{=0} + P(E) = 1 + P(E) > 0$.
\end{enumerate}
\end{enumerate}
\end{itemize}
\end{prue}

\begin{coro}[Unicidad de escritura de las fórmulas]
Si $\alpha$ es una fórmula entonces se verifica uno y sólo uno de las
siguientes casos.

\begin{enumerate}
\item $\alpha$ es una variable proposicional.
\item Existe una única fórmula $\beta$ tal que $\alpha = \neg \beta$.
\item Existe un único conectivo binario $*$ y únicas fórmulas $\beta_1, \beta_2$
tales que $\alpha = (\beta_1 * \beta_2)$.
\end{enumerate}
\end{coro}

\begin{prue}.
\begin{itemize}
\item Si ocurre 1) no hay nada que probar.
\item Si ocurre 2) $\alpha = \neg \beta = \neg \beta'$ en este caso vale $\beta = \beta'$
\item Sino $\alpha = (\beta_1 * \beta_2) = (\gamma_1 \circ \gamma_2)$ con 
$\beta_1, \beta_2, \gamma_1, \gamma_2$ fórmulas y $*,\circ \in \{\vee, \wedge, \Rightarrow\}$.

Queremos probar que $\beta_1 = \gamma_1, * = \circ, \beta_2 = \gamma_2$. Simplificando
los paréntesis se obtiene $\overbrace{\beta_1}^{E} \overbrace{*\  \beta_2}^{F} = 
\underbrace{\gamma_1}_{G}\underbrace{\circ\ \gamma_2}_{H}$.

Hay tres casos posibles

\begin{enumerate}[label=\alph*)]
\item $\ell(E) = \ell(G)$, en este caso $E = G, * = \circ, F = H$, de donde se obtiene
lo que queriamos probar.
\item $\ell(E) > \ell(G)$, en este caso $\beta_1 = \gamma_1 \circ \_$ por la parte 2)
de la proposición anterior $P(\gamma_1) > 0$, absurdo pues $\gamma_1$ es fórmula y
esto contradice la parte 1) de la proposición anterior $P(\gamma_1) = 0$.
\item $\ell(G) > \ell(E)$, al igual qu eel caso b) se obtiene un absurdo.
\end{enumerate}
\end{itemize}
\end{prue}

\begin{defn}
Sea $\alpha$ una fórmula diremos que $\beta$ es una subfórmula de $\alpha$ de acuerdo
a los siguientes casos:
\begin{enumerate}
\item $\alpha$ es una variable proposicional $P_i$, en este caso $\beta$ es una subfórmula
de $\alpha$ si y sólo si $\beta = \alpha$.
\item $\alpha = \neg\gamma$ con $\gamma$ fórmula (única), en este caso $\beta$ es subfórmula
de $\alpha$ si y sólo si $\beta = \alpha$ ó $\beta$ es subfórmula de $\gamma$.
\item $\alpha = (\gamma_1 * \gamma_2)$ con $* \in \{\vee, \wedge, \Rightarrow\}$, 
$\gamma_1, \gamma_2$ fórmulas. $\beta$ es subfórmula de $\alpha$ si y sólo si 
$\beta = \alpha$ ó $\beta$ es subfórmula de $\gamma_1$ ó $\beta$ es subfórmula
de $\gamma_2$.
\end{enumerate}
\end{defn}

\begin{exap}
$\alpha = (\underbrace{\neg P_1}_{\gamma_1} \vee \underbrace{(P_2 \Rightarrow P_3)}_{\gamma_2})$.

Subfórmulas de $\alpha$: $P_1, P_2, P_3, (P_2 \Rightarrow P_3), \neg P_1, \alpha$.
\end{exap}

\begin{ejer}
Si $X_1, X_2, \dots, X_n$ es una cadena de formación de una fórmula $\alpha$ y 
$\beta$ es subfórmula de $\alpha$ entonces existe $i \in \{1,2,\dots,n\}$
tal que $\beta = X_i$
\end{ejer}

\begin{prob}
Si $X_1, X_2, \dots, X_n$ es una cadena de formación minimal de uan fórmula $\alpha$
¿Es cierto que $X_i$ es subfórmula de $\alpha$ para $1 \le i \le n$?
\end{prob}

\section*{Semántica de la lógica proposicional}

$Lenguaje\ S = \{V \leftarrow V + 1, V \leftarrow V - 1, \text{if } V \neq 0 \text{ goto }L\}$.

$A = Variables\ Proposicionales \cup \{\vee, \wedge, \Rightarrow, \neg\} \cup \{(,)\}$.

\begin{defn}
$2 = \{0, 1\}$, 2 es un conjunto ordenado $0 < 1$, 0: ``falso'', 1: ``verdadero''. Otras
alternativas $\{F, V\}$, $\{\bot, \top\}$.
\end{defn}

\subsection*{Interpretación de los conectivos}

\begin{itemize}
\item ``negación'': $\neg \longrightarrow \neg : 2 \to 2,\ \neg(x) = 1 - x$
\item ``conjunción'': $\wedge \longrightarrow \wedge : 2 \to 2,\ \wedge(x, y) = \min(x,y)$
\item ``disyunción'': $\vee \longrightarrow \vee : 2 \to 2,\ \vee(x, y) = \max(x, y) = x + y - xy$
\item ``implicación'': $\Rightarrow \longrightarrow \Rightarrow : 2 \to 2,\ 
\Rightarrow(x, y) = \max(1 -x, y)$
\end{itemize}




\end{document}
